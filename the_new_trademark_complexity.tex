\documentclass[letterpaper, 11pt, oneside]{article}
\usepackage[mmddyyyy]{datetime}
\usepackage{setspace}
\usepackage[utf8]{inputenc}
\usepackage{mathptmx}
\usepackage{titlesec}
\usepackage{ragged2e}
\usepackage{fancyhdr}
\usepackage[all]{nowidow}
\usepackage{url}
\usepackage{tocloft}
\usepackage[headheight = 110pt]{geometry}
\usepackage{indentfirst}
\setlength{\parindent}{25pt}
\usepackage{tikz}

\urlstyle{same}
\PassOptionsToPackage{hyphens}{url}\usepackage[hidelinks]{hyperref}

\makeatletter
\renewcommand\normalsize{\@setfontsize\normalsize{12}{14}}
\renewcommand\small{\@setfontsize\small{10}{12}}
\renewcommand\scriptsize{\@setfontsize\scriptsize{7}{8}}
\renewcommand\tiny{\@setfontsize\tiny{5}{6}}
\renewcommand\large{\@setfontsize\large{13}{15}}
\renewcommand\Large{\@setfontsize\Large{14}{16}}
\renewcommand{\thesection}{\Roman{section}}
\renewcommand{\thesubsection}{\Alph{subsection}}
\renewcommand{\thesubsubsection}{\arabic{subsubsection}}
\renewcommand{\theparagraph}{\alph{paragraph}}
\renewcommand{\thesubparagraph}{\roman{subparagraph}}
\titleformat{\section}{\normalfont\normalsize\scshape\center}{\thesection.}{1em}{}
\titleformat{\subsection}{\normalfont\itshape\center}{\thesubsection.}{1em}{}
\titleformat{\subsubsection}{\normalfont}{\thesubsubsection.}{1em}{}
\titleformat{\paragraph}{\normalfont}{\theparagraph.}{1em}{}
\titleformat{\subparagraph}{\normalfont\small}{\thesubparagraph.}{1em}{}
\titlespacing{\subparagraph}{0pt}{12pt}{12pt}
\makeatother

\renewcommand\contentsname{\hspace*{\fill}\normalfont\normalsize\scshape Table of Contents\hspace*{\fill}}
\renewcommand\cftsecfont{\normalfont\normalsize\scshape}
\renewcommand\cftsubsecfont{\normalfont\normalsize\itshape}
\renewcommand\cftsubsubsecfont{\normalfont\normalsize}
\renewcommand\cftparafont{\normalfont\normalsize}
\renewcommand\cftsubparafont{\normalfont\small}
\renewcommand\cftsecpagefont{\normalfont\normalsize}
\renewcommand{\cftsecleader}{\cftdotfill{\cftdotsep}}
\renewcommand\abstractname{}

\setcounter{secnumdepth}{5}
\setcounter{tocdepth}{5}

\cftsetindents{section}{3em}{3em}
\cftsetindents{subsection}{4em}{3em}
\cftsetindents{subsubsection}{5em}{3em}
\cftsetindents{paragraph}{6em}{3em}
\cftsetindents{subparagraph}{7em}{3em}

\fancypagestyle{firstpage}{
\fancyhf{}
\fancyhead[R]{\textit{Draft -\ \today}}
\renewcommand{\headrulewidth}{0pt}
}

\fancypagestyle{subsequentpages}{
  \fancyhf{}
  \fancyhead[L]{\thepage}
  \fancyhead[C]{\uppercase{New Trademark Complexity}}
  \fancyhead[R]{\textit{In-process draft -\\ \today \\ Not for distribution, etc.}}
  \renewcommand{\headrulewidth}{0pt}
}

\usepackage{enumitem}
\setlist{  
  listparindent=\parindent,
  parsep=0pt,
}

\usepackage{graphicx}
\graphicspath{{./graphics/}}
\usepackage{caption}
\usepackage{subcaption}
\usepackage{float}

\usepackage{xcolor}
\definecolor{seafoam}{HTML}{099999}
\definecolor{oxfordblue}{HTML}{203769}

\title{\Large{\uppercase{The New Trademark Complexity: An Empirical Analysis of Design Search Codes at the USPTO}}}
\author{\large\textit{Lucas Daniel Cuatrecasas}\thanks{Furman Fellow, New York University School of Law. [Acknowledgments forthcoming.]}}
\date{\textit{\small [In-process draft -\ \today]}}

\begin{document}
\maketitle\thispagestyle{firstpage}\pagenumbering{gobble}

\begin{abstract}

\begin{itshape}

Although the prototypical trademark may be a brand name (e.g., APPLE, NIKE), trademark law also protects a vast array of nonverbal product features (e.g., the physical appearance of an Apple Store, the design of Nike Air Jordan shoes). Trademark doctrine largely conceives of all these things in the same way: as symbols that tell the consuming public who makes a product. But that conception is becoming increasingly untenable. Drawing on data culled from 10.7 million trademark applications filed at the U.S. Patent \& Trademark Office (USPTO), this Article presents evidence of rising numbers of applications for nonword trademarks that are highly specific, intricate, and often unlikely to be recognized by consumers who are not familiar with the nuances of a particular industry. In so doing, this Article provides the first large-scale empirical analysis of the ``design search codes'' the USPTO assigns to any trademark application containing images. Since the USPTO initiated its modern design coding system in the mid-1980s, the number of applications for highly complex nonword trademarks, as measured by number of design search codes, has risen at least 700\%. The most complex marks have also increased as a proportion of all applications, even though total trademark applications have dramatically increased during that time period. One explanation for this is that having more codes increases a mark's odds of USPTO approval. Another explanation is that, as desirable word marks have become more scarce, businesses have begun to substitute away from words into images.

There are two principal reasons to be concerned about this trend. First, as highly specific and detailed nonword trademarks become more common, there is a growing risk that some of these nonverbal marks—often those applied for by luxury brands—effectively subsidize a feeling of in-group membership for the consumers that can recognize those marks, a dynamic that reinforces patterns of social inequality. Second, opportunistic or even fraudulent applicants, incentivized by the favorable odds of USPTO approval for complex nonword marks, may prosecute nonword applications that they wouldn't have otherwise pursued, resulting in ``clutter'' on the USPTO's registers that shrinks the public domain and may disadvantage competing businesses. In response, this Article proposes that considering consumer sophistication as part of the threshold inquiry into a nonword trademark's validity can help temper the potential overprotection of highly complex nonword marks, as well as the risk that marks recognized by sophisticated consumers will receive too broad a scope of protection.

\end{itshape}

\end{abstract}

\tableofcontents

\pagestyle{subsequentpages}\pagenumbering{arabic}
\addtocounter{page}{1}

\addcontentsline{toc}{section}{Introduction}
\section*{Introduction}\label{sec:0}

If asked to think of a trademark, one's first thought is probably a name—say, APPLE, NIKE, or UBER.\footnote{Indeed, it's difficult to draw a clear line between the abstract idea of a brand and the specific name that represents that brand in consumers' minds—this may be a distinction without a difference. \textit{Cf.} Alexander Chernev, Ryan Hamilton \& David Gal, \textit{Competing for Consumer Identity: Limits to Self-Expression and the Perils of Lifestyle Branding}, 75 \textsc{J. Mktg.} 66, 69–70 (2011) (reporting results of an experiment in which experimenters asked participants to think of ``brands'' significant to themselves or their parents and subsequently asked participants to express their preference for certain \textit{brand names}, with the null hypothesis being that thinking of personally-significant brands (as opposed to brands significant to others) in the first step would not significantly reduce the strength of the preferences articulated in the second step).} And justly so. Trademarks refer to a product's source, and as the Supreme Court has recently reminded us, the prototypical trademark is the name of a product's maker.\footnote{\label{supra1} Vidal v. Elster, 602 U.S. 286, 305–06 (2024) (quoting Ainsworth v. Walmsley, L.R. 1 Eq. 518, 525 (1866)) (``[I]s not a man’s name as strong an instance of trade-mark as can be suggested?'').} ``[E]ven as late as 1860 the term `trademark' really denoted only the name of the manufacturer."\footnote{\textit{Id.} at 305 (quoting Beverly W. Pattishall, \textit{Two Hundred Years of American Trademark Law}, 68 \textsc{Trademark Rep.} 121, 128 (1978) (quoting a treatise)).} In some ways, not much has changed. Today, the words that represent the brand value of trillion- or hundred-billion-dollar companies like Apple, Nike, and Uber are simply modern avatars of the ancient practice of identifying products by the verbal symbols etched onto them.\footnote{\textit{See} \textsc{Google Finance}, \url{https://www.google.com/finance/} (providing the market capitalizations of all publicly listed U.S. companies); \textit{infra} Section~\ref{sec:2}.\ref{subsec:2C}.}

Yet this view of trademarks is also somewhat cramped. Nonverbal features of products—colors, shapes, sounds—can be trademarks too.\footnote{\textit{See generally} \textsc{The Protection of Non-Traditional Trademarks: Critical Perspectives} (Irene Calboli \& Martin Senftleben eds., 2018) [hereinafter \textsc{The Protection of Non-Traditional Trademarks}].} Consider the wrapper of a Hershey’s Kiss, the physical appearance of an Apple store, or the design of Nike Air Jordan shoes. All are federally registered trademarks.\footnote{Registration No. 6,639,129 (design of upper panels on Air Jordan shoes); Registration No. 4,021,593 (façade of Apple's Fifth Avenue, New York City location); Registration Nos. 4,277,914, 4,277,913 (interior layout of Apple Stores in general); Registration No. 186,828 (a single wrapped Hershey's Kiss).} And trademark law largely takes for granted that these nonword trademarks represent a continuation of the historical association between names and trademarks.\footnote{\textit{See infra} Section~\ref{sec:2}.\ref{subsec:2C}.} But does that make sense? As scholars have increasingly emphasized, trademark law's fixation on words risks overlooking the nuances of nonword trademarks, which can evoke in consumers' minds a broad range of connotations, some of which may be untethered from trademarks' core function of denoting a product's source.\footnote{\label{supra3} \textit{See} Dustin Marlan, \textit{Visual Metaphor and Trademark Distinctiveness}, 93 \textsc{Wash. L. Rev.} 767, 769, 772 (2018) (``We live in an age where visual images dominate commerce and branding, yet trademark law fails to properly analyze image marks . . . .''); Rebecca Tushnet, \textit{Looking at the Lanham Act}, 48 \textsc{Hous. L. Rev.} 861, 866 (2011) (``Trademark law has struggled to fit nonword marks into the [test for evaluating word marks] and has failed to offer any particularly well-formed alternatives to [that test] for images or other nonverbal symbols that might indicate source, with the partial exception of product packaging trade dress.''); \textit{id.} at 912–14 (discussing how trade dress can make nonverbal claims about products' healthfulness or environmental impact); \textit{cf.} Mark A. Lemley \& Mark P. McKenna, \textit{Trademark Spaces and Trademark Law’s Secret Step Zero}, 75 \textsc{Stan. L. Rev.} 1, 12–13 (2023) (noting that the category of image marks that are not traditional logos ``has never been particularly well-defined'').} Indeed, commercial images can often ``encode" complex, multifaceted meanings in a way words can't.\footnote{Tushnet, \textit{supra} note [X], at 898 n.156 (quoting William D. Wells, \textit{Three Useful Ideas}, 13 \textsc{Advances in Consumer Res.} 9, 9 (1986) (discussing how in advertisements ``casting, lighting, accents, dialogue, music, body language—all the instruments of drama'' ``encode[]'' information about a product in addition to the express claims made about that product)).} Given that ability to encode, would it be surprising if businesses used nonword trademarks in a way that has little to do with the model of trademarks as names? In fact, what if businesses used nonword trademarks less as \textit{names} and more as \textit{passwords}—a visual code known only by the initiated?

This Article argues that this is increasingly the case. Drawing on data culled from 10.7 million trademark applications filed at the U.S. Patent \& Trademark Office (USPTO), I present evidence of rising numbers of applications for nonword trademarks that are highly specific, intricate, and often unlikely to be recognized by consumers who are not familiar with the nuances of a particular industry. In so doing, I provide the first large-scale empirical analysis of the ``design search codes'' the USPTO assigns to any trademark application containing images. Design search codes are a series of six-digit sequences, each of which corresponds to a specific type of design that may appear in a mark (e.g., 01.01.01 corresponds to stars with three points). Each significant visual element in a mark receives such a code (e.g., Mercedes's logo \includegraphics[scale = 0.1]{RN_7122938_drawing} would receive that code, as well as a code for a circle). Take, for example, the particular background pattern on the boxes containing Caperdonich Speyside single-malt whisky: \par \

\begin{figure}[H]
\centering
\begin{subfigure}[h]{0.2\linewidth}
\includegraphics[width = \linewidth]{SN_98056114_drawing} \
\caption{Drawing of mark submitted to USPTO}
\end{subfigure}
\hspace{30pt}
\begin{subfigure}[h]{0.225\linewidth}
\includegraphics[width = \linewidth]{SN_98056114_specimen} \
\caption{Specimen of mark use}
\end{subfigure}
\caption*{Serial No. 98,056,114}
\end{figure}
\par

\noindent Or the shape of the particular key clasp used on the ``Réjane Nano'' handbag sold by the understated French luxury fashion brand Moynat: \par \

\begin{figure}[H]
\centering
\begin{subfigure}[h]{0.4\linewidth}
\includegraphics[width = \linewidth]{RN_5139501_drawing} \
\caption{Drawing of mark submitted to USPTO}
\end{subfigure}
\hspace{30pt}
\begin{subfigure}[h]{0.3\linewidth}
\includegraphics[width = \linewidth]{RN_5139501_specimen} \
\caption{Specimen of mark use}
\end{subfigure}
\caption*{Registration No. 5,139,501}
\end{figure}
\par

\noindent What these USPTO-approved marks have in common is that they are highly complex, based on their numerous design search codes. Since the USPTO initiated its modern design coding system in the mid-1980s, the number of applications for such highly complex nonword trademarks has risen at least 700\% and has also increased as a proportion of all applications, suggesting meaningful—and potentially troubling—changes to our aesthetic environment. This trend is particularly striking given that nonword applications, as a group, have actually \emph{decreased} as a proportion of all applications over the same period.

One explanation for this is that applications for highly complex marks reflect a strategy by applicants to better their chances of USPTO approval. I find support for that explanation: a regression analysis indicates that having more codes increases a mark's odds of USPTO approval. But that's not the whole story. Over the past few decades, economic growth, the rise of Amazon's brand registry, and applicant opportunism have come together to generate a pronounced rise in the overall number of applications for word marks filed at the USPTO.\footnote{\textit{See infra} Section~\ref{sec:2}.\ref{subsec:2B}.\ref{subsubsec:2B1}.} And while I do not argue that this increase in trademark filings has \emph{caused} the proportion of highly complex nonword marks to increase, the data and their context suggest that the two trends are meaningfully intertwined. In particular, given recent scholarship finding that unclaimed but desirable word marks are becoming scarcer,\footnote{\label{supra4} \textit{See} Barton Beebe \& Jeanne C. Fromer, \textit{Are We Running out of Trademarks?}, 131 \textsc{Harv. L. Rev.} 945 (2018) (showing that a high proportion of words that would be desirable trademarks—namely, frequently-used English words, single-syllable words, and common surnames—are either already registered as marks or are confusingly similar to an already-registered mark).} one might infer that businesses are substituting or diversifying away from applications for word marks and into applications for nonword marks, as images may be less depleted than words. Indeed, the data indicate that applications for highly complex marks have increased as a proportion of all applications most sharply within several categories of products where word marks have become especially depleted, which may suggest such a substitution or diversification effect.

Higher numbers of complex nonword marks create two principal problems, particularly against the background of increasing numbers of trademark filings. First, rising nonword mark complexity threatens to push trademark law out of sync with market reality, exacerbating consumer inequality. If subtle branding elements like the Caperdonich box pattern and Réjane clasp are trademarks, then, in theory, those branding elements serve to tell consumers who makes the relevant product. Yet these branding elements are strikingly understated, context-specific, and nuanced—a far cry from the instantly-recognizable word NIKE or the iconic façade of Apple's Fifth Avenue store. If anything, an ability to recognize nonword marks like the Caperdonich box pattern or Réjane clasp seems more like a sign of connoisseurship than an automatic result of these marks' presence in the market. Indeed, it's not for nothing that, in the fashion world, the aesthetic surrounding such branding elements has recently taken on the name of ``stealth wealth'' or ``quiet luxury.'' Where it was once clear and in-your-face, many businesses' branding—particularly that of luxury businesses—has become more complex and recherché, recognizable only by the initiated.\footnote{\textit{See, e.g.}, Gretha Cachia, \textit{The Challenges of Protecting Quiet Luxury Brands}, \textsc{Glob. Legal Post} (June 20, 2024), \url{https://www.globallegalpost.com/news/the-challenges-of-protecting-quiet-luxury-brands-1414585052} (``It’s a new aesthetic of understated affluence and elegance for refined and educated tastes. . . . Quiet luxury’s aura of mystery is precisely what creates a sense of educated exclusivity in the elite social circles of people who can recognise (and afford) premium possessions.''); Guy Trebay, \textit{What's the Status of Flaunting Your Status?}, \textsc{N.Y. Times} (July 22, 2023), \url{https://www.nytimes.com/2023/07/22/style/quiet-luxury-wealth-status.html} (``[F]or those in the class of New Old Money — that is, great fortunes made, often in tech, in a time frame bracketed by Myspace and TikTok — wealth display is noticeable, but only to those who know what they’re looking for.''); Jessica Dickler, \textit{At a Time When Most Americans Are Living Paycheck to Paycheck, the ‘Quiet Luxury’ Trend Takes Over}, CNBC (June 10, 2023), \url{https://www.cnbc.com/2023/06/10/quiet-luxury-may-be-americans-most-expensive-trend-to-date.html} (suggesting that the ``stealth wealth'' or ``quiet luxury'' aesthetic is exemplified by outfits worn by actress Gwyneth Paltrow during her courtroom appearances in connection with ski tort litigation in Utah).} Yet trademark doctrine is not equipped to understand this distinction. Trademark doctrine takes for granted that, even if consumers may not have previously encountered a mark (e.g., the word mark CAPERDONICH on a bottle), \textit{they will nonetheless understand that it is a mark}. Perversely, this means that the new trademark complexity may co-opt intellectual property protection of nonword marks to subsidize a feeling of in-group membership for certain consumers—at the expense of trademark law's core goal of reducing search costs for \textit{all} consumers.

Second, as noted, the more complex a nonword mark is, the better its odds of receiving USPTO approval. That's because elevated complexity often makes nonword marks sufficiently unique or uncommon to pass the test for ``distinctiveness,'' the key requirement for any trademark.\footnote{\textit{See infra} Part~\ref{sec:1}.\ref{subsec:1B}.} But this dynamic threatens to be tremendously wasteful. Prosecuting each one of these marks requires a nontrivial amount of USPTO resources—indeed, more resources than word marks do.\footnote{\textit{See infra} Part~\ref{sec:1}.\ref{subsec:1C}.} Those resources are misdirected if USPTO-approved trademarks are unlikely to have source-indicative meaning to consumers beyond a select few. Moreover, the favorable odds of obtaining approval for a visually complex mark may incentivize applicants to pursue opportunistic or even fraudulent applications for nonword marks that, all else equal, they would not have pursued.\footnote{\textit{See} Barton Beebe \& Jeanne Fromer, \textit{Fake Trademark Specimens: An Empirical Analysis}, 120 \textsc{Colum. L. Rev. F.} 217 (2020) (finding, among other things, that in a random sample of 365 trademark applications filed at the USPTO in 2017 66.9\% included fraudulent specimens, meaning that the applicants were not in fact using those marks in commerce, as required by law).} This can result in marks that appear active on the USPTO's registers but are not actually being used by their ostensible owners. Such trademark ``clutter'' reduces the quantity of images in the public domain and disadvantages competing businesses who wish to use similar marks.\footnote{\textit{See} Beebe \& Fromer, \textit{supra} note~\ref{supra4}, at 951 n.19, 1034 (``Cluttering refers to marks that are registered but not used in commerce by their registrants in one or more of the classes in which they are registered.'').} Indeed, just as recent scholarship shows that good \textit{word} trademarks are becoming more scarce, we could potentially see a similar problem with nonword trademarks in the long term.\footnote{Barton Beebe and Jeanne Fromer have suggested that image marks are potentially becoming more depleted (in the sense that fewer images that do not conflict with existing marks are available) and congested (in the sense that a larger number of similar images are being claimed by different firms for the same products), just as is occurring with word marks. \textit{See id.} at 951, 1041.}

To head off those bad outcomes, this Article argues that the threshold test for nonword trademark validity—the analysis of a trademark's \textit{distinctiveness} to consumers—should include an express consideration of ``consumer sophistication.'' Generally treated as a measure of how much consumers are paying attention in a mark's specific commercial context, consumer sophistication is currently a factor in the test for trademark \textit{infringement}. In theory, more attentive consumers can more easily spot a copycat. Greater sophistication can thus weigh against infringement.\footnote{\label{supra5} \textsc{J. Thomas McCarthy, McCarthy on Trademarks and Unfair Competition} \S\S 23:95 (5th ed. 2024) (``If the goods or services are relatively expensive, more care is taken and buyers are less likely to be confused as to source or affiliation.''); \textit{see} Thomas R. Lee, Glenn L. Christensen \& Eric D. DeRosia, \textit{Trademarks, Consumer Psychology, and the Sophisticated Consumer}, 57 \textsc{Emory L.J.} 575, 579–83 (2008) (discussing the consumer sophistication factor and noting that courts generally follow the view that ``a sophisticated consumer is one who is apt to spend more time, attention, or care in making a purchasing decision—and who is thus deemed less likely to be confused as to the source or sponsorship of the trademarked products she buys''). The distinctiveness analysis is different from the infringement analysis. The focus of the distinctiveness analysis is whether a mark has source-indicative meaning to consumers, and the focus of the infringement analysis is, classically, whether consumers are likely to be confused into thinking that the defendant's products are produced by or affiliated with the plaintiff. The former analysis precedes the latter in court opinions, since there can be no infringement without a valid mark, which must be distinctive. \textit{See, e.g.} Gruner + Jahr USA Publ. v. Meredith Corp., 991 F.2d 1072, 1075 (2d Cir. 1993) (``We begin analysis with the first step — whether plaintiff's mark merits protection, before proceeding to the second step, the likelihood of confusion.'').} I contend that consumer sophistication should figure into nonword mark distinctiveness in the same way. That is, in scenarios where consumers are unlikely to pay close attention, courts and the USPTO should require particularly strong evidence of distinctiveness before deeming a nonword mark valid, especially where that nonword mark differs from a conventional logo (e.g., is a highly detailed background pattern). This would guard against opportunistic exploitations of trademark complexity that could lead to trademark clutter. Conversely, in scenarios where consumers are likely to pay close attention—such as with luxury products\footnote{\textit{See, e.g.}, Hamilton Int'l Ltd. v. Vortic LLC, 13 F.4th 264 (2d Cir. 2021) (approvingly noting the distinct court's finding that the four-figure price of defendant's wristwatches, among other facts, indicated that the relevant consumers would be sophisticated); Cartier, Inc. v. Deziner Wholesale, L.L.C., No. 98 Civ. 4947, 2000 WL 347171, at *6 (S.D.N.Y. Apr. 3, 2000) (``Cartier customers, as consumers of luxury items, are presumed to be sophisticated buyers who can and will distinguish between Deziner and Cartier sunglasses, despite any confusing references on Deziner's packages.'')}—courts and the USPTO may rely on visual complexity itself as evidence of distinctiveness, just as trademark law currently permits. But with one caveat: consistent with other areas of trademark doctrine, stronger evidence of distinctiveness should entitle a mark to a broader scope of protection.\footnote{The positive correlation between distinctiveness and scope of protection manifests itself in the consideration of a mark's strength as part of the infringement analysis. \textsc{McCarthy}, \textit{supra} note~\ref{supra5}, \S 23:40.50 (``The stronger the senior user's mark, the greater the range of permutations of junior users' marks that can trigger a likelihood of confusion.''); \textit{see also} Barton Beebe \& C. Scott Hemphill, \textit{The Scope of Strong Marks: Should Trademark Law Protect the Strong More than the Weak?}, 92 \textsc{N.Y.U. L. Rev.} 1339, 1341, 1393 (2017) (discussing the ``blackletter doctrine'' that, in infringement analyses, a mark's scope of protection ``monotonic[ally]'' increases with its distinctiveness but arguing for a general rejection of that doctrine in the absence of facts that make its application compelling). This correlation also manifests itself in the powerful ``antidilution'' protection available to marks that are ``widely recognized by the general consuming public of the United States.'' 15 U.S.C. \S 1125(c) (providing that the owner of such a mark may enjoin another's use of a mark or tradename that is likely to dilute the owner's mark by impairing its distinctiveness or harming its reputation).} And weaker evidence, a narrower scope. Thus, even where consumer sophistication makes it easier for a business to claim a nonword mark, the scope of that mark's protection should be narrow unless the business provides evidence, beyond mere complexity, of the mark's distinctiveness to consumers. This more nuanced distinctiveness test would be more aligned with market reality and less likely to overprotect highly complex nonword marks that are in tension with trademark law's policy goals.

This Article proceeds in three Parts. Part~\ref{sec:1} briefly discusses the framework for federal trademark registration and the key requirement of trademark distinctiveness. This Part then explains how the USPTO uses design search codes to categorize nonword trademarks. It also reviews the formative process by which these codes grew out of the USPTO's historical digitization initiatives. Part~\ref{sec:2} presents my empirical analysis. It begins with an explanation of how design search codes can reflect a mark's complexity. This Part then provides evidence that, in recent decades, the number and proportion of applications for highly complex nonword marks has risen. I suggest that some applicants may be turning to highly complex nonword marks in an attempt to secure a trademark registration, given the favorable odds of approval of such applications, as well as the dearth of unclaimed, desirable word marks. Market trends—like a shift toward recherché expressions of luxury—may also account for higher numbers of visually complex marks to some extent. I then argue that trademark law's current relationship to nonverbal complexity likely results in overprotection of highly complex nonword marks, which risks amplifying socioeconomic inequality and creating trademark clutter. Part~\ref{sec:3} suggests that one workable—though hitherto underexplored—option for mitigating that overprotection is to consider consumer sophistication as part of the threshold inquiry into a nonword trademark's validity. A brief Conclusion follows.

\newpage

\section{Background}\label{sec:1}

\subsection{The Federal Trademark Registration System}\label{subsec:1A}

The Lanham Act is the principal source of modern American trademark law. One of its core functions is the regulation of federal trademark registration.\footnote{\label{supra6} \textit{See} Rebecca Tushnet, \textit{Registering Disagreement: Registration in Modern American Trademark Law}, 130 \textsc{Harv. L. Rev.} 867, 941 (2017) (generally calling for greater scholarly attention to the registration system established by the Lanham Act and specifically observing that ``given the number of registrations compared to the far smaller number of infringement lawsuits, they’re not the primary creator of the legal rights in the trademark system. The registration system is . . . .'').} There are two trademark registers: the principal register and the supplemental register.\footnote{\label{supra7} 15 U.S.C. \S 1052 (setting forth the requirements for principal register registration); 15 U.S.C. \S 1091 (``All marks capable of distinguishing applicant’s goods or services and not registrable on the principal register provided in this chapter, except those [that are otherwise barred from registration or are not in use in commerce,] may be registered on the supplemental register upon the payment of the prescribed fee and [proper submission of an application].'').} The principal register contains marks that are ``distinctive,'' a concept explained farther below, and the supplemental register contains marks that have not yet achieved distinctiveness but are capable of doing so.\footnote{\label{supra8} \textit{See} Tushnet, \textit{supra} note~\ref{supra6}, at 878, 882, 885–892 (explaining how a mark's distinctiveness has become coextensive with its registrability on the principal register); \textit{supra} note~\ref{supra7}.} For reasons discussed below, the principal register is more important and more desirable to applicants—accordingly, unless otherwise specified, references to trademark registration in this Article refer to the principal register. Per its statutory obligations, the USPTO makes the principal and supplemental registers available to the public through what is essentially an ``internet database'' of marks.\footnote{35 U.S.C. \S 41(i)(1) (``The Director [of the USPTO] shall maintain, for use by the public, paper, microform, or electronic collections of . . . United States trademark registrations arranged to permit search for and retrieval of information.''); \textit{In re} Tam, 808 F.3d 1321, 1347 (Fed. Cir. 2015), \textit{aff'd sub nom.} Matal v. Tam, 137 S. Ct. 1744 (2017). Anyone can search the entirety of the principal and supplemental registers, along with all applications filed at the USPTO since the mid-1980s, by using the USPTO's online ``Trademark Search'' tool. \textit{Trademark Search}, USPTO, \url{https://tmsearch.uspto.gov/search/search-information}.}

A trademark doesn't need to be on either of these registers to receive protection against infringement.\footnote{\textit{E.g.}, Iancu v. Brunetti, 588 U.S. 388, 391 (2019) (``Registration of a mark is not mandatory. The owner of an unregistered mark may still use it in commerce and enforce it against infringers.''); Tushnet, note~\ref{supra6}, at 878–80.} But registration entails certain important advantages. A registration provides constructive notice of a business's ownership of the mark in question.\footnote{15 U.S.C. \S 1072.} Further, registered trademarks enjoy a presumption of validity in litigation.\footnote{\textit{Id.} \S 1057(b).}  This makes sense, as the inquiry into whether a mark is valid in litigation is the same as the inquiry into whether that mark is entitled to federal registration.\footnote{\textit{See supra} note~\ref{supra8}.} A similar carveout exists for the rule that nonword trademarks are invalid if they are ``functional''—i.e., if they employ product features that improve a product's engineering or aesthetic usefulness.\footnote{\textit{See infra} Section~\ref{sec:1}.\ref{subsec:1B}.} Registered nonword trademarks are free from the burden of proving their nonfunctionality in litigation.\footnote{\S 1125(a)(3).}  Other advantages of registration include nationwide priority and eligibility to become incontestable. The former means that a registrant is deemed to have been using their mark nationwide as of the date they filed their trademark application, even if they weren't actually doing so as of that date.\footnote{\S 1057(c)} The latter means that, after five years of consecutive use, a registrant may apply to immunize their mark from certain challenges to its validity.\footnote{\label{supra8} \S 1065. Perhaps the most important—and most controversial—benefit of incontestability is that it forecloses the argument that a word mark is ``merely descriptive'' of the relevant product and therefore not distinctive. Tushnet, note~\ref{supra6}, at 903 n.160; Barton Beebe, \textit{Is the Trademark Office a Rubber Stamp?}, 48 \textsc{Hous. L. Rev.} 751, 755 (2011); \textit{infra} Section~\ref{sec:1}.\ref{subsec:1B}.}

Moreover, the USPTO encourages—and lawyers will routinely engage in—searching the registers for marks that are similar to a business's proposed mark before that business files an application.\footnote{\label{supra9} \textit{E.g.}, Amy A. Abeloff, \textit{Clearing Trademarks: Back to Basics with Practical Tips and Tricks}, 11 \textsc{Landslide} 56, 58 (2019) (urging lawyers to ``[a]lways recommend clearing a mark'' to clients); \textit{Why Search for Similar Trademarks?}, USPTO, \url{https://www.uspto.gov/trademarks/basics/why-search-similar-trademarks}.} Thus, in addition to providing constructive notice, a registration may also provide actual notice, discouraging others from adopting similar marks.\footnote{Abeloff, \textit{supra} note~\ref{supra9}, at 57 (explaining how searching the registers may lead to such notice).} To a large extent, this applies to marks on the supplemental register too.\footnote{Tushnet, note~\ref{supra6}, at 922 n.238.} Similarly, mark owners may find that citing their own registrations in cease-and-desist letters, or license demands, addressed to potential infringers can lend those communications additional gravitas.\footnote{\textit{E.g.}, \textit{The Benefits of Trademark Registration}, \textsc{Law Offices of Derek A. Simpson}, \url{https://dsimpsonlegal.com/benefits-trademark-registration/} (``Someone started using a mark confusingly similar to yours? The cease-and-desist letters you or your lawyers will need to write carry much more weight when they cite a federal registration.''). Unfortunately, it's not only legitimate mark owners that benefit here. Rent-seeking ``trademark trolls'' that collect registrations for unused marks in order to harass others with threats of litigation can benefit in equal measure.\textit{See} Angela Peterson, \textit{Overdue Notice: Using Virtual Marking to Modernize Trademark Notice Requirements}, 25 \textsc{Stan. Tech. L. Rev.} 247, 257 (2022) (reviewing the literature on this phenomenon).}

The USPTO makes the initial determination to approve an application to register a mark.\footnote{35 U.S.C. \S 1 (establishing the USPTO); \textit{infra} note~\ref{infra1}.} Applicants may submit a use-based application, for marks that they are currently using, or they may submit an ``intent to use'' (ITU) application, for marks that they have a bona fide intent to use within up to three years following the USPTO's approval of their mark.\footnote{\label{infra1}; 15 U.S.C. \S 1051(a)(1) (providing for use-based applications); \textit{id.} \S 1051(b)(1) (providing for ITU applications); \textit{id.} \S 1051(d)(1)–(2) (three-year timeframe for ITU applications).} Both use-based and ITU applications must set forth the proposed mark and a description of its nonverbal elements, if any, and the goods and services for which the proposed mark is used (or, for ITU applications, for which the proposed mark is to be used), among various other pieces of information.\footnote{15 U.S.C. \S 1051(a)(2), (b)(2); 37 CFR \S\S 2.32, 2.37 (``A description of the mark must be included if the mark is not in standard characters.'').} The categorization of goods/services is based on the Nice Classification, which was established by the 1957 Nice Agreement, an international treaty, and which contains 45 separate, broadly-articulated ``international classes'' of goods/services into which products may fall (e.g., ``Electrical and scientific apparatus''). Applicants must also pay a relatively low fee, which increases linearly with each additional class of goods or services claimed in the application.\footnote{37 CFR \S 2.6.}

As Section~\ref{sec:1}.\ref{subsec:1C} discusses in detail, upon receiving an application, the USPTO will assign ``design search codes'' to any significant nonword elements of the proposed mark.\footnote{\label{supra10} Coding of Design Marks in Registrations, 75 Fed. Reg. 81587 (Dec. 28, 2010) (``When an application with design elements is filed, specially trained Federal employees in the Pre-Examination section of the USPTO review the mark drawing and assign USPTO Design Classification codes.'').} A USPTO examining attorney will then review the application to determine if it meets the requirements for registration.\footnote{\textit{Id.}} An analysis of the mark's distinctiveness is key to this determination.\footnote{\textit{See supra} note~\ref{supra8}.} For nonword marks, the USPTO may also inquire into functionality, among other potential issues.\footnote{\textit{See infra} Section~\ref{sec:1}.\ref{subsec:1B}.} If the mark meets the requirements, it will ultimately be registered so long as no one opposes the mark's registration during a 30-day window to do so (and, in the case of ITU applications, so long as the applicant has timely submitted proof of their use of the mark).\footnote{15 U.S.C. \S\S 1051(d)(1), 1052, 1063(a).} If the mark doesn't meet the requirements, the examining attorney will issue an ``office action,'' explaining where the proposed mark falls short.\footnote{37 C.F.R. \S 2.61(a). Final office actions can be appealed to the Trademark Trial and Appeal Board (TTAB), the appellate body of the USPTO. TTAB decisions can, in turn, be appealed to the Federal Circuit or any other federal district court with jurisdiction. 15 U.S.C. \S 1070 (appeals to TTAB); 15 U.S.C. \S 1071(a)(1), (b)(1) (appeals from TTAB); 37 C.F.R. \S 2.141(a) (appeals to TTAB); 37 C.F.R. \S 2.145(a)(1), (c)(1) (appeals from TTAB). However, the USPTO is only bound by Federal Circuit precedent. \textsc{U.S. Pat. \& Trademark Off., U.S. Trademark Trial and Appeal Board Manual of Procedure (TBMP)} \S 101.03 (2024), \url{https://www.uspto.gov/sites/default/files/documents/tbmp-Master-June2024.pdf} (listing the adjudicators whose precedential decisions govern the TTAB and stating that ``[t]he [TTAB] relies primarily on precedent from the Court of Appeals for the Federal Circuit'').}

Registrations must be renewed every 10 years.\footnote{15 U.S.C. \S 1059(a).} Additionally, an affidavit of continuing use of the mark is due between the fifth and sixth anniversaries of the registration date and then again at the beginning of each renewal period.\footnote{15 U.S.C. \S 1058(a).} However, unlike other forms of intellectual property protection (e.g., copyrights, design patents), trademark protection is not time-limited. Theoretically, a trademark registration's life can be infinite.\footnote{\textit{E.g.}, Beebe, \textit{supra} note~\ref{supra8}, at 754 (noting that a registration ``may be renewed in perpetuity''); \textit{see also} Jennifer Jenkins, \textit{Mickey, Disney, and the Public Domain: a 95-year Love Triangle}, \textsc{Duke L.: Ctr. for Study of Pub. Domain} (2024), \url{https://web.law.duke.edu/cspd/mickey/} (``If Amazon goes on using the `Amazon' brand for 500 years, the trademark stays alive.'').} 

\subsection{Trademark Distinctiveness} \label{subsec:1B}

Distinctiveness is the sine qua non of trademark protection. If a mark is distinctive, that means that consumers will perceive it as an indicator of the source of a particular product.\footnote{\textit{See} 15 U.S.C. \S 1127 (requiring any mark to ``identify and distinguish [the owner's] goods, including a unique product, from those manufactured or sold by others and to indicate the source of the goods, even if that source is unknown''); \textit{supra} note~\ref{supra8}.} This source-indicativeness is conventionally regarded as the fundamental role of any trademark.\footnote{\textsc{McCarthy}, \textit{supra} note~\ref{supra5}, \S 3:1 (``The role that a designation must play to become a `trademark' is to identify the source of one seller's goods and distinguish that source from other sources. If the designation performs that role, then the law deems it to be `distinctive' and legally protectable.'').} However, as Parts~\ref{sec:2} and~\ref{sec:3} will explore, nonword trademarks may in practice diverge from that role—perhaps increasingly so.

Like much of trademark law, modern distinctiveness doctrine grew out of categories that originally sought to describe words. Words can be distinctive in two ways. ``Inherently distinctive'' words are words that the law presumes consumers automatically recognize as indicating a commercial source.\footnote{\textit{Id.} \S 11:4 (``An inherently distinctive designation is presumed to immediately serve as an identifier of source from the very first moment that it is used as a mark.'').} Under the well-known ``\textit{Abercrombie} spectrum''—named after the seminal decision in \textit{Abercrombie \& Fitch Co. v. Hunting World, Inc.}—three types of words are inherently distinctive: (1) ``arbitrary'' words, which are unrelated to the relevant product (e.g., STARBUCKS for coffee), (2) “fanciful” words, which are neologisms (e.g., EXXON for gas), and (3) ``suggestive'' words, which only connect to the relevant product with some imagination (e.g., LYFT for a ridesharing service).\footnote{\label{supra11} Abercrombie \& Fitch Co. v. Hunting World, Inc., 537 F.2d 4 (2d Cir. 1974) (Friendly, J.); Starbucks Corp. v. Wolfe's Borough Coffee, Inc., 736 F.3d 198, 212 (2d Cir. 2013) (STARBUCKS is arbitrary); Exxon Corp. v. XOIL Energy Res., Inc., 552 F. Supp. 1008, 1014 (S.D.N.Y. 1981) (EXXON is fanciful); Christopher Buccafusco, Jonathan S. Masur \& Mark P. McKenna, \textit{Competition and Congestion in Trademark Law}, 102 \textsc{Tex. L. Rev.} 437, 442 (2024) (LYFT is suggestive).} In contrast, under \textit{Abercrombie}, words that are ``merely descriptive'' of the product (e.g., ``FOR WALKING'' for footwear) are not inherently distinctive and thus must have ``acquired distinctiveness'' to be protectable as trademarks.\footnote{Registration No. 7077829 (consisting of trademark rights to the words ``FOR WALKING'' (\textit{with} those quotation marks) for footwear, on the basis of acquired distinctiveness and subject to a disclaimer of the words FOR WALKING (\textit{without} quotation marks)).} Words have acquired distinctiveness when consumers have learned that those words indicate a commercial source, even if those words are not inherently distinctive.\footnote{\textit{See, e.g.}, Wal-Mart Stores, Inc. v. Samara Bros., 529 U.S. 205, 211 (2000) (``[A] mark has acquired distinctiveness, even if it is not inherently distinctive, if it has developed secondary meaning, which occurs when, `in the minds of the public, the primary significance of a [mark] is to identify the source of the product rather than the product itself.'\thinspace'' (quoting Inwood Lab'ys, Inc. v. Ives Laby's, Inc., 456 U.S. 844, 851 n.11 (1982))); \textit{see also} Buccafusco, Masur \& McKenna, \textit{supra} note~\ref{supra11}, at 451 (explaining that acquired distinctiveness results from a mark's ``secondary meaning—secondary not in the sense of being of secondary importance, but in the sense of being second in time to the primary, descriptive meaning'').} Acquired distinctiveness can be shown through direct evidence (e.g., consumer surveys) or circumstantial evidence (e.g., advertising expenditure).\footnote{\textsc{McCarthy}, \textit{supra} note~\ref{supra5}, \S 15:70.} However, ``generic'' words for a product (e.g., COFFEE for coffee) are, as a matter of law, never distinctive, no matter how much acquired distinctiveness they may have. After all, if only one business could use the generic term for a product, its competitors would be at a disadvantage, because they would not be able to directly indicate what they sell.

Trademark law has long since extended the concept of distinctiveness to a wide range of nonverbal matter. When a mark is principally or entirely nonverbal, it's often referred to as ``trade dress,'' a term that can refer to ``a product’s packaging or configuration as well as nearly any other aspect of the product or service.''\footnote{\textsc{Barton Beebe, Trademark Law: An Open-Source Casebook} 130 (version 11, digital ed. 2024), \url{https://www.tmcasebook.org/wp-content/uploads/2024/07/BeebeTMLaw-v11-digital-edition-2024.pdf}.} As with words, only certain categories of trade dress can be inherently distinctive. In particular, trade dress that is ``product packaging'' (i.e., ``the three-dimensional packaging or wrapping in which a product is sold'') may be inherently distinctive.\footnote{\label{supra12} \textit{Wal-Mart}, 529 U.S. at 212; \textsc{U.S. Pat. \& Trademark Off., Trademark Manual of Examining Procedure} (TMEP) \S 1202.02(f)(ii).} So can the ``ambiance'' or ``look and feel'' associated with a business's services (e.g., the décor of a restaurant).\footnote{The Supreme Court has referred to this type of trade dress as a ``\textit{tertium quid}'' that is not quite product packaging but doesn't constitute product design either. \textit{Wal-Mart}, 529 U.S. at 212 (giving restaurant décor as an example of this type of trade dress). Dustin Marlan has dubbed this type of trade dress ``service dress'' and has argued that, contrary to current doctrine, service dress should never be deemed inherently distinctive. Dustin Marlan, Tertium Quid \textit{Unveiled: Trade Dress and Service Design}, 58 \textsc{U.C. Davis L. Rev.} (forthcoming 2024/2025).} However, trade dress that is ``product design'' (i.e., those features that make a product ``more useful or appealing'') must have acquired distinctiveness to be protectable.\footnote{\textit{Wal-Mart}, 529 U.S. at 213; TMEP, \textit{supra} note~\ref{supra12}, \S 1202.02(f)(i).} The same goes for any trade dress that consists of a single color alone.\footnote{Qualitex Co. v. Jacobson Prods. Co., 514 U.S. 159, 163–64 (1995). However, the Federal Circuit has held that \textit{multiple} colors, when used on \textit{product packaging}, may be inherently distinctive. \textit{In re} Forney Indus., 955 F.3d 940 (Fed. Cir. 2020). Squaring \textit{Forney} with \textit{Qualitex} is not straightforward. Lemley \& McKenna, \textit{supra} note~\ref{supra3}, at 23 (pointing out that \textit{Forney} seems to ignore that \textit{Qualitex} expressly held that color alone can never be inherently distinctive).} If trade dress is eligible to be inherently distinctive, then the USPTO and most courts will determine whether it is inherently distinctive by applying the test set forth in \textit{Seabrook Foods, Inc. v. Bar-Well Foods, Ltd}. This test asks whether the trade dress (1) is a ```common' basic shape or design,'' (2) is ``unique or unusual in a particular field,'' (3) is a ``mere refinement of a commonly-adopted and well-known form of ornamentation for a particular class of goods viewed by the public as a dress or ornamentation for the goods,'' and (4) can produce ``a commercial impression distinct from [any] accompanying words.''\footnote{Seabrook Foods, Inc. v. Bar-Well Foods, Ltd., 568 F.2d 1342, 1344 (C.C.P.A. 1977). In an illustration of the long shadow language has cast on trademark law's evolution, some courts have analyzed the inherent distinctiveness of trade dress by drawing on the \textit{Abercrombie} spectrum, either on its own or in combination with a \textit{Seabrook}-like analysis. \textit{See} \textsc{McCarthy}, \textit{supra} note~\ref{supra5}, \S 8:13 (citing, among other cases, Fun-Damental Too, Ltd. v. Gemmy Industries Corp., 111 F.3d 993, 1000 (2d Cir. 1997) (applying \textit{Abercrombie} to product packaging); Ashley Furniture Indus., Inc. v. SanGiacomo N.A. Ltd., 187 F.3d 363, 371 (4th Cir. 1999) (reading \textit{Seabrook} as an ``elaboration of \textit{Abercrombie}'').} As for acquired distinctiveness, the evidence used to show acquired distinctiveness of trade dress is the same as it is for word marks.\footnote{\textsc{McCarthy}, \textit{supra} note~\ref{supra5}, \S 8:8:50 (``The traditional methods of proving validity through the acquisition of a secondary meaning in non-inherently distinctive word marks are also used to prove secondary meaning for trade dress.'').}

But although this is the blackletter law, the contours of the trade dress–distinctiveness analysis are in practice quite fluid. For example, on occasion, the USPTO seems to ignore the product design/packaging distinction and analyze the inherent distinctiveness of product features that could well be viewed as product design.\footnote{\label{supra13} The Moynat Réjane handbag clasp discussed in the Introduction is an example of a product feature that the USPTO appears not to have viewed as product design (since it was \textit{not} registered on the basis of inherent distinctiveness), even though the clasp is not packaging or wrapping and seems clearly to be a product feature that renders the bag ``more useful or appealing.'' Registration No. 5,139,501. Similarly, in some cases the USPTO has analyzed the inherent distinctiveness of back-pocket designs on jeans, although the Federal Circuit has held that decorative features on the back of jeans are product design. \textit{Compare In re} Right-On Co., 87 U.S.P.Q.2d 1152 (T.T.A.B. 2008) (precedential), \textit{with In re} Slokevage, 441 F.3d 957 (Fed. Cir. 2006); \textit{see also} Lucas Daniel Cuatrecasas, \textit{Failure to Function and Trademark Law's Outermost Bound}, 96 \textsc{N.Y.U. L. Rev.} 1312, 1324 n.82 (2021).} Likewise, as I have explained elsewhere, the USPTO applies \textit{Seabrook} inconsistently, often adding to or subtracting from the test's factors stated above.\footnote{\textit{See} Cuatrecasas, \textit{supra} note~\ref{supra13}, at 1340.} More generally, courts and the USPTO will often start their inherent distinctiveness analysis by listing out the four \textit{Seabrook} factors but proceed to narrowly focus on the single question of whether the mark is unique or uncommon in context.\footnote{\textit{E.g.}, Yankee Candle Co., Inc. v. Bridgewater Candle Co., LLC, 259 F.3d 25 (1st Cir. 2001); \textit{In re} The Procter \& Gamble Co., 105 U.S.P.Q.2d (BNA) 1119 (T.T.A.B. 2012) (precedential). To shore up their narrow approach, decisions of this kind generally rely on treatise writer Thomas McCarthy's reading of \textit{Seabrook}: ``In essence, the elements of the \textit{Seabrook} test are merely different ways to ask whether the design, shape or combination of elements is so unique, unusual or unexpected in this market that one can assume without proof that it will automatically be perceived by customers as an indicator of origin—a trademark.'' \textsc{McCarthy}, \textit{supra} note~\ref{supra5}, \S 8:13.50.} It is probably for this reason that, as I show in Part~\ref{sec:2}, a nonword mark's visual complexity is a statistically significant predictor of the USPTO's approval of that mark.\footnote{The concept of ``generic'' words has a counterpart in the context of nonword marks, illustrating again how the rules for word marks and the rules for nonword marks are interwoven. Nonword marks that are ``functional'' can never be trademarks, no matter how distinctive they are. \textsc{McCarthy}, \textit{supra} note~\ref{supra5}, \S 7:63. There are two types of functionality: utilitarian and aesthetic. \textit{E.g.}, Christian Louboutin S.A. v. Yves Saint Laurent Am. Holdings, Inc., 696 F.3d 206, 217 (2d Cir. 2012) (``The `functionality' of a mark can be demonstrated by, \textit{inter alia}, showing that the mark has either traditional `utilitarian' functionality or `aesthetic' functionality.''). Under the doctrine of utilitarian functionality, ``a product feature is functional if it is essential to the use or purpose of the article or if it affects the cost or quality of the article''—an inquiry that focuses on the engineering usefulness of a nonword mark (e.g., its ability to make a product more durable or easy to use). Inwood Laby's, Inc. v. Ives Laby's., Inc., 456 U.S. 844, 850 n. 10 (1982); Apple, Inc. v. Samsung Electronics Co. Ltd., 786 F.3d 983 (Fed. Cir. 2015). In contrast, under the doctrine of aesthetic functionality, no product feature is protectable if ``trademark rights [in that feature] would significantly hinder competition.'' Qualitex Co. v. Jacobson Prods. Co., 514 U.S. 159, 170 (1995) (quoting \textsc{Restatement (Third) of Unfair Competition} § 17, comment c (\textsc{Am. L. Inst.} 1995)). This inquiry instead focuses on whether a nonword mark's appearance itself is useful (e.g., a nonword mark is a color that looks especially good when applied to the kind of product at issue). Brunswick Corp. v. British Seagull Ltd., 35 F.3d 1527, 1531 (Fed. Cir. 1994). The idea that some nonword marks are too generally useful to be privatized parallels the doctrine of genericism to a striking degree. Indeed, the USPTO has denied protection to certain nonword marks on the ground of genericism when (had it been raised by the parties) functionality could also have been a ground for that denial. \textit{E.g.}, Milwaukee Elec. Tool Corp. v. Freud Am., Inc., Cancellation Nos. 92,059,634 \& 92,059,637 (T.T.A.B. Dec. 2, 2019) (precedential) (cancelling, on the basis of genericism, two registrations for the color red applied to a saw blade, one of which had initially received, but overcome, a rejection from the USPTO on the basis of functionality); 15 U.S.C. \S 1064(3) (providing that registrations may be cancelled on the grounds of, among others, genericism and functionality).}

\subsection{Design Search Codes at the USPTO}\label{subsec:1C}

Design search codes are a classification system maintained by the USPTO. These codes seek to organize the theoretically infinite range of visual elements that could be part of a trademark.\footnote{\label{supra14} Request for Comments on Removal of Paper Search Collection of Marks That Include Design Elements, 71 Fed. Reg. 36065 (June 23, 2006) (``The design codes cover all of the possible designs that can be put into a trademark application and are used to search design marks.'').} In so doing, they attempt to help both trademark examiners and applicants in identifying preexisting marks that may conflict with a proposed mark.

This classification system is detailed in the USPTO's design search code manual.\footnote{\textit{Trademark Design Search Code Manual}, USPTO, https://tmdesigncodes.uspto.gov/.} Each design search code has six digits. The first two digits are its category, the next two are its division, and the last two are its section.\footnote{\label{supra15} \textit{Trademark Design Search Code Manual - Introduction and General Guidelines}, USPTO, \url{https://tmdesigncodes.uspto.gov/} [hereinafter \textit{Design Search Code Guidelines}].} There are 29 sequentially-numbered categories. The first 28 categories consist of a general and somewhat arbitrary type of design, whereas category 29 consists of colors and small, inconspicuous design elements functioning as punctuation or parts of letters (e.g., the ``i" dotted with a flame in \includegraphics[scale = 0.15]{RN_2052508_drawing}).\footnote{Registration No. 2052508.} Each category contains 13 or fewer divisions, all of which are odd-numbered and represent more specific manifestations of the general category. Finally, each division contains 38 or fewer sections, which differentiate between visual elements at the lowest level of granularity.\footnote{With respect to the codes in category 29 that represent colors, the divisions represent different configurations of one or multiple colors, and the sections represent specific colors. \textit{Trademark Design Search Code Manual - Design Search Codes}, USPTO, https://tmdesigncodes.uspto.gov/.}

Thus, for example, the USPTO will assign to the mark \includegraphics[scale = 0.225]{RN_4820434_drawing}\footnote{Registration No. 4820434.} the code 03.01.04, indicating that the mark falls into section 04 (``domestic cats'') within division 01 (``cats, dogs, wolves, foxes, bears, lions, tigers'') of category 03 (``animals'').\footnote{Often, however, a division or section is a catchall group defined in reference to the preceding divisions or sections. For example, a code beginning with 08.01 corresponds to a design of ``baked goods'' within the category of ``foodstuff.'' The codes from 08.01.01 to 08.01.12 all correspond to various specific baked goods (e.g., ``coffee cake,'' ``bagels''), but the code 08.01.25 is for ``miscellaneous baked goods'' other than the prior specific baked goods.} Each ``significant design element[]'' in a mark with visual elements receives a code—i.e., the more significant elements, the more codes.\footnote{\textit{Design Code Guidelines}, \textit{supra} note~\ref{supra15} (instructing coders to code ``significant design elements'').} For example, the Apple logo, \includegraphics[scale = 0.10]{RN_1114431_drawing}, \footnote{Registration No. 1114431.} has only one design search code: the code for apples (05.09.05). But the Starbucks logo, \includegraphics[scale = 0.125]{RN_4538053_drawing}, \footnote{Registration No. 4538053.} has four codes corresponding to each of its significant elements: a star with five points (01.01.03), a mermaid (04.03.03), a crown that is open at the top (24.11.02), and a circle that is totally or partially shaded (26.01.21). Thus, as Section~\ref{sec:2}.\ref{subsec:2A} will discuss further, design search codes provide a rough but good measure of how visually complex a given nonword mark is, because the number of codes a mark has reflects how many significant elements it has.

The USPTO's design search codes are based on the ``Vienna Classification,'' established by the 1973 Vienna Agreement Establishing an International Classification of the Figurative Elements of Marks.\footnote{Vienna Agreement Establishing an International Classification of the Figurative Elements of Marks, June 12, 1973, 1863 U.N.T.S. 317.} The United States is not a member of this international treaty.\footnote{\textit{WIPO Lex Database Search}, WIPO, https://www.wipo.int/wipolex/en/main/treaties.} However, the United States does participate in the larger international-law framework that this treaty is a part of—a system that, among other things, makes it possible to apply for a single ``international registration'' for a mark, which is in effect a registration in the national offices of countries within this framework.\footnote{In particular, the United States is a member of the 1883 Paris Convention for the Protection of Industrial Property and the 1989 Madrid Protocol relating to the 1891 Madrid Agreement Concerning the International Registration of Marks, both of which are administered by the World Intellectual Property Association (WIPO), a specialized United Nations agency. \textit{Id.}; \textit{see} \textsc{McCarthy}, \textit{supra} note~\ref{supra5}, \S\S 29:25, 29:32. Among other important rights, the Paris Convention gives trademark applicants who file an initial application in one member country a six-month period during which the filing date for that first application will be applied to a new application in another member country. As an alternative, the Madrid Protocol permits mark owners to apply for trademark registrations in multiple member countries by making one international application, which can mature into an international registration. Paris Convention for the Protection of Industrial Property, art. 4, Mar. 20, 1883, 21 U.S.T. 1583, 828 U.N.T.S. 305; Protocol Relating to the Madrid Agreement Concerning the International Registration of Marks, art. 2, June 27, 1989, 828 U.N.T.S. 389, https://www.wipo.int/wipolex/en/text/283483.} International registrations for marks with visual elements employ the Vienna Classification, and many national trademark offices in addition to the USPTO do so as well.\footnote{\textit{See} \textsc{Madrid Legal Div., Guide to the Madrid System: International Registration of Marks Under the Madrid Protocol} 85 (2022), \url{https://www.wipo.int/edocs/pubdocs/en/wipo-pub-455-2022-en-guide-to-the-international-registration-of-marks-under-the-madrid-protocol.pdf}; \textit{Frequently Asked Questions: Vienna Classification}, WIPO, \url{https://www.wipo.int/classifications/vienna/en/faq.html} (``Around 60 offices in the world apply the Vienna Classification. This number includes member as well as non-member countries.'').} Thus, the Vienna Classification facilitates transferability of information between international and national mark databases, as well as between national mark databases in different countries.

The USPTO's decision to adopt the Vienna Classification is also interwoven with the USPTO's decades-long efforts to transition to fully digital recordkeeping—a process that has culminated in the wealth of digital data the USPTO currently makes available to the public. Before 1983, the USPTO maintained its records in paper form.\footnote{\label{supra16} \textit{See} Thomson \& Thomson v. Quigg, 10 U.S.P.Q.2D (BNA) 1741, 1743 n.12 (D.D.C. 1989) (discussing the 1983 agreements whereby Thomson \& Thomson, TCR Services, Inc., and Compu-Mark agreed to digitize various aspects of the USPTO's trademark files in exchange for access to and use of the data in those files); \textsc{Am. Bar Ass'n, Section of Pat., Trademark \& Copyright L., 1983 Summary of Proceedings} app.A (1983) (discussing the USPTO's initial digitization of trademark status and location data and its ongoing efforts at comprehensive digitization of USPTO trademark data).} Examining attorneys and the public would have to search those records in ``tall cabinets located at the USPTO's Public Search Facility,'' although there existed private search services that maintained digitized records of USPTO trademark data.\footnote{Corsearch, Inc. v. Thomson \& Thomson, 792 F. Supp. 305, (S.D.N.Y. 1992) (indicating that, by the early 1980s, Thomson \& Thomson, Inc. had fully digitized its private records of federal and state trademarks); Coding of Design Marks in Registrations, \textit{supra} note~\ref{supra10}, at 81587; \textit{see also} Susan Trausch, \textit{Names Are No Game: What's in a Name? Often Millions of Dollars}, \textsc{Boston Globe}, Oct. 28, 1980, at 34 (``Working primarily with corporate lawyers, Thomson \& Thomson does between 20,000 and 30,000 searches a year for an average price of \$100 each.'').} However, in 1980, Congress passed legislation requiring that the USPTO create, within two years, a plan for digitizing patent and trademark prosecution data.\footnote{Bayh-Dole Act, Pub. L. No. 96-517, \S 9.} In response, the USPTO engaged several private search firms to help it digitize its trademark records.\footnote{\textit{See Thomson \& Thomson}, 10 U.S.P.Q.2D (BNA) at 1743; \textsc{Am. Bar Ass'n, Section of Pat., Trademark \& Copyright L.}, \textit{supra} note~\ref{supra16}, app.A; \textit{see also} Policy and Plans Regarding Exchange Agreements; Competitive Procurement Contracts; Electronic Patent Data Disseminations Guidelines, 50 Fed. Reg. 49980 (Dec. 6, 1985) (discussing the terms of the USPTO's 1983 digitization agreements with Thomson \& Thomson and Compu-Mark).} As part of this effort, one firm assigned design search codes to visual elements in the digitized versions of all trademark registrations and applications that were ``active'' as of 1983.\footnote{\textit{See Thomson \& Thomson}, 10 U.S.P.Q.2D (BNA) at 1743 (``T\&T was to create and develop a computerized database of registered trademarks and servicemarks consisting of designs or images, commonly known as `logos,' [and] assign descriptive design codes to these logos . . . .'').} The USPTO then assigned search codes to visual elements in all trademark applications going forward.\footnote{The USPTO historically maintained a secondary coding system for visual elements of marks, the ``Trademark Search Facility Classification Code Index.'' Coding of Design Marks in Registrations, \textit{supra} note~\ref{supra10}, at 81587. This system was more rudimentary than the design search code system—in particular, it categorized designs based only on general categories, without the additional refinement offered by the design search code system. 75 FR 81587; 76 FR 11431 (describing the design search code system as ``much more specific, precise and robust''). Further, this system was originally available only in the USPTO's paper files, and, at least as of 2001, the USPTO coded only trademark registrations, not applications, with this system. https://www.uspto.gov/web/offices/com/sol/og/2016/week52/TOCCN/item-455.htm After making this secondary system available electronically for only three and a half years, the USPTO discontinued this system shortly before the USPTO ceased its ongoing maintenance of legacy paper records and transitioned those records to microform format. 76 FR 11431} Consistent with their origin in the USPTO's digitization project, design search codes have always been available only through the USPTO's online database.

Thus, under today's system, USPTO staff manually assign design search codes to each new application with visual elements before the USPTO reaches an initial decision on the proposed mark's registrability.\footnote{73 FR 13780} Applicants receive notice of the codes assigned to their application and may suggest changes or additions to the codes.\footnote{71 FR 36065} However, the USPTO has the ultimate authority to determine the appropriate codes for each application.\footnote{73 FR 13780} Importantly, codes have no \textit{substantive} legal significance.\footnote{73 FR 13780} However, they have practical significance, since, during the examination process, trademark examiners will search for marks with the same codes. That can lead to discovery of preexisting registrations or applications that the examiner thinks are confusingly similar to the proposed mark. That results in an office action rejecting the mark. Additionally, as discussed, registration of a mark is constructive notice of its ownership by the registrant and may provide actual notice of that ownership too. Because design search codes may also help potential applicants identify registrations or applications that could be confusingly similar to their proposed mark, the design search codes assigned to an existing mark may mean the difference between a potential applicant applying for their preferred mark (if they are not able to identify a conflicting mark) and a potential applicant choosing a different mark (if they are able to identify a conflicting mark).

Design search codes are far from perfect. One can nitpick them endlessly. For example, one might fairly ask why the State of Texas enjoys its own specific design search code (01.17.12), while all other States must share one design search code (01.17.11); why ``equipment for animals'' is placed, together with ``transport'' and ``traffic signs,'' in its own category (category 18), as opposed to folding equipment for animals into the category for tools (category 14); and why a distinction must be drawn between a single drop of liquid (01.15.08) and multiple drops (01.15.09), while no such distinction is drawn between one single snowflake and multiple snowflakes (01.15.09). Indeed, Rebecca Tushnet has likened design search codes' classification scheme to the absurd taxonomy of animals contained in Argentine writer Jorge Luís Borges's essay, ``The Analytical Language of John Wilkins,'' which categorizes animals into groups like ``fabulous ones'' and ``those that resemble flies from a distance.''\footnote{Barton Beebe has made a similarly apt observation with respect to the Nice Classification.}

Moreover, design search codes are subject to human error. After the USPTO digitized its records in the '80s, Congress passed legislation permitting the USPTO to eliminate its physical records, subject to a public notice-and-comment process and to presenting Congress with a plan for pursuing such elimination without public harm. But after presenting such a plan to Congress, the USPTO received many public comments expressing concern about that plan because of, among other things, errors in the USPTO's digital records. In particular, several commenters identified errors in the design search codes assigned to marks, with a study cited in the comments showing a 52\% error rate in a sample of marks with design search codes. The most common type of error was a failure to assign a code to a portion of a design, but assignment of the wrong code to a particular design (e.g., assigning the code for a palm tree to a design of a dog) was also common. Similarly, commenters also noted that some marks with visual elements lacked design search codes entirely. The USPTO's response to this public concern was multifaceted, and a full analysis is beyond the scope of this Article. However, with respect to design search codes, the USPTO made corrections to historical design search code assignments to the extent it agreed that they were erroneous, increased efforts to monitor the quality of the design search codes assigned to applications, and, in 2007, made multiple changes to the design search code system aimed at increasing its specificity.\footnote{https://www.uspto.gov/web/offices/com/sol/og/2016/week52/TOCCN/item-455.htm Additionally, in 2006, the USPTO also conducted a randomized study of newly filed applications and concluded ``that only 4.5\% of records contained errors relating to significant elements of a mark that would negatively impact the ability to retrieve such a mark during a search for confusingly similar marks.'' The USPTO believed this reflected improvement, because a USPTO study of applications filed in 2001-2002 had found a 19\% coding error rate. However, the USPTO's changes did not rectify all historical design search code errors, and the potential for erroneous coding remains. https://www.uspto.gov/web/offices/com/sol/og/2016/week52/TOCCN/item-455.htm}

But, despite their flaws, there are no signs that the USPTO intends to overhaul the design search code system itself—at least not in the short term. After all, a new system would require not only changing design coding practices going forward but also recoding the USPTO's entire archive of active marks, given that searches for potentially conflicting marks must cover any historical registration that remains in use and any historical application that has not yet been rejected. Similarly, because the USPTO's design search codes are based on the Vienna Classification, they are interoperable with the classifications used by other trademark offices around the world, and coordination costs could prevent swift agreement on a better system. Additionally, advances in machine learning may (somewhat ironically) further ingrain the USPTO's use of design search codes that are ultimately selected by a human. The USPTO has recently been experimenting with the use of a machine learning model that, based on a training dataset of prior examiner decisions, will make an initial evaluation of the design search codes that an examiner should assign to a new proposed mark. The examiner will then use their judgment to add to or subtract from the marks the model initially chooses. Such partial automation may augment design coding's longevity as a system by increasing the efficiency with which examiners can assign codes.

\section{An Empirical Analysis of Design Search Codes at the USPTO}\label{sec:2}

In this Part, I show that applications for visually complex nonword marks are becoming increasingly common at the USPTO, evincing meaningful and potentially troubling aesthetic changes.

Before beginning this Part's analysis, a word on scope is in order. This Part's arguments rely on an analysis of the USPTO's publicly available ``Trademark Case Files'' dataset, which contains data on around 12.7 million trademark applications filed at the USPTO. Technical details about this dataset are included in Appendix A. The core of my analysis concerns the 332,568 applications filed from 1986 through 2023 that the USPTO has categorized as containing no words or letters—i.e., purely visual applications. Because applications for word marks are dramatically more common, these 332,568 applications represent about 3\% of the roughly 10.7 million trademark applications filed at the USPTO from 1986 through 2023. Thus, by focusing only on purely visual marks, this Part necessarily focuses on a very small subset of the hundreds of thousands of applications filed at the USPTO each year. In all likelihood, the trends occurring within that small subset will be quite different from trends occurring outside of it. Indeed, as some of the examples in the Part suggest, filings for complex, specific trade dress are often the province of established brands seeking to protect designs in markets for highly differentiated and/or luxury products (think back to the Réjane handbag clasp discussed in the Introduction). Nonword mark filings may thus seem like a rarefied area of trademark law, detached from the bigger, more ubiquitous world of verbal marks—a world where emerging brands are just as likely to stake a claim in their name as are the most well-known brands in the world (e.g., APPLE).

But that is exactly the point. Nonword marks are subject to unique pressures. As Section~\ref{sec:2}.\ref{subsec:2C} argues, one of those pressures may in fact be the ballooning quantity of \textit{word} mark filings. Perhaps equally important, though, is the historical interconnectedness between the evolution of nonword marks and luxury brands' intellectual property strategies—strategies that often aim to maintain an effective monopoly on a nonverbal product feature. This formative influence of high-end, consumer-facing brands means that nonword marks may often have an outsized effect in shaping consumers' aesthetic environment and in preserving markets for luxury goods, which depend on the exclusion of certain consumers. Put otherwise, it's not just that nonword marks may reflect and amplify the trends occurring among word marks. Nonword marks are also a bellwether of aesthetic change—and, particularly, aesthetic change tied to broader patterns of socioeconomic exclusion. Zooming in on nonword marks provides a particularly clear vignette of how the trademark system undergirds and facilitates those patterns.

\subsection{Design Search Codes as an Indicator of Complexity}\label{subsec:2A}

As noted above, the empirical analysis presented in this Part assumes that a mark's design search codes can indicate how visually complex that mark is. At a high level of generality, this proposition is self-evident. If a mark has more codes, that means it has more ``significant design elements.'' Definitions of visual complexity across academic disciplines associate a design's complexity with how many different elements it has—effectively, how ``busy'' it is.\footnote{\label{supra24} \textit{See, e.g.}, Helena Miton \& Olivier Morin, \textit{Graphic Complexity in Writing Systems}, 214 \textsc{Cognition} 1, 2, 4, 12 (2021) (observing that, in writing systems, complex characters ``tend to involve a greater number of distinct strokes'' and defining character complexity using perimetric complexity (i.e., $\frac{P^2}{4{\pi}A}$, where $P$ is the perimeter of the inked surface and $A$ is the surface area) and algorithmic complexity (essentially a measure of the minimum size of a computer file that can encode the character)); Michael Bauerly \& Yili Liu, \textit{Effects of Symmetry and Number of Compositional Elements on Interface and Design Aesthetics}, 24 \textsc{Intl. J. Human-Comput. Interaction} 275, 285 (reporting, in the context of an experiment regarding subjective reactions to interface aesthetics, that participants reacted negatively to ``high number of elements (rectangles or Web page images), which is directly related to the complexity of the image''); R.P. Taylor, B. Spehar, C-W.G. Clifford \& B.R. Newell, \textit{The Visual Complexity of Pollock's Dripped Fractals}, \textit{in} \textsc{Unifying Themes in Complex Systems} 175, 177 (Ali A. Minai \& Yaneer Bar-Yam 2008) (explaining that ``[a]n important parameter for quantifying a fractal pattern's visual complexity is the fractal dimension, D,'' a real number between 1 and 2 that increases as a fractal pattern becomes more ``intricate'' and ``detailed''); \textit{cf.} \textsc{Don Norman, Living with Complexity} 60 (2010) (``Sparse, clean designs have an aesthetic appeal, but . . . they may be more difficult to use than crowded complex-looking designs, where many different alternatives and options are always on display.''). } Thus, it stands to reason that the more codes a design has, the more complex it is.

But design search codes are an imperfect indicator of visual complexity, despite that core relationship to how busy a mark is. This is due to three interrelated particularities of the design coding system:

\begin{enumerate}

\item[a.] \textit{Overcoding bias}. The USPTO generally errs on the side of too many, rather than too few, codes. If more than one design search code could apply to the same design element, coders will assign \emph{all} applicable codes.\footnote{\label{supra25} \textit{Design Search Code Guidelines}, \textit{supra} note~\ref{supra15} (guidelines 6, 7, 10). This principle sometimes undergirds the USPTO's guidance on design coding in ways that are not obvious. For example, if a mark contains multiple instances of the same element, and separate codes exist for \emph{a single instance of that element} and for \emph{multiple instances of that element}, then the mark is ``double coded'' with both the single-element code and the multiple-element code. \textit{See id.} After all, both of those codes could apply to multiple instances of the same thing.} For example, a mark that depicts the head of an animal (take \includegraphics[scale = 0.225]{RN_4820434_drawing}\footnote{Registration No. 4820434.} as an example again) will receive both the general code for depictions of that animal (here, the code for cats, 03.01.04) and the specific code for depictions of \emph{heads} of that animal (the code for heads of cats, 03.01.18). This double-codes the same visual element.\footnote{Moreover, when a design has several significant elements, coders must assign a code to \emph{each} of those elements, even when one of those codes already covers more than one of the significant elements. \textit{See id.} (guideline 5) (``Where design elements contain distinctive component parts, both the design element as a whole and its distinctive component(s) are coded.''). For example, if presented with the proposed mark \includegraphics[scale = 0.2]{RN_2741681_drawing}, coders must assign it \emph{both} the code for humans playing instruments (02.09.15) \emph{and} the code for guitars (22.01.06), even though the former code already indicates the presence of an instrument. Although this aspect of the USPTO's guidance does not necessarily lead to a number of codes that is greater than the number of significant elements, this is also a form of duplicative coding, because it applies multiple codes to the same element.} Similarly, ``[i]f there is doubt as to whether a particular design element should be coded,'' USPTO examiners are instructed to ``code the design element in question.''\footnote{\textit{See id.} (guideline 6).}

Coders are also instructed to code visual elements in applicants' mark drawings \emph{even where applicants expressly disclaim those elements}. Such elements are typically shown in dashed lines. For example, adidas's registration for the mark \includegraphics[scale = 0.35]{RN_3029127_drawing},\footnote{Reg. No. 3029127.} which consists of adidas's three-stripe trade dress on a garment, expressly disclaims the garment on which the mark is shown, because adidas is claiming the design \emph{on} the garment, not the garment itself. Yet in addition to the design search code for straight lines (26.17.01), the USPTO also assigned this mark the design search code for jackets (09.03.01). This makes sense from the perspective of providing actual notice to potential infringers. The disclaimed matter is essential, because it is the commercial context in which adidas uses the mark. However, from the perspective of measuring mark complexity, the presence of two codes suggests that there are two significant elements in the mark when, arguably, we should focus only on the claimed matter, the three stripes, which amount to one element (or perhaps three distinct elements).\footnote{\label{supra26} Indeed, per its own guidance, the USPTO should actually have coded the adidas three-stripe mark for \emph{both} three lines \emph{and} a single line. \textit{See supra} note~\ref{supra25}.} Viewed from this angle, coding disclaimed matter contributes to a general tendency to overcode, rather than undercode. The upshot is that the relationship between a mark's significant design elements and the number of design search codes it has is far from perfectly linear. 

\item[b.] \textit{Lack of relationship between codes and constituent points, lines, or shapes}. The USPTO's overcoding bias raises a second issue. Even if does reflect the \emph{number of significant visual elements} in a mark, the number of codes assigned to a mark does not always bear a direct relationship to each \emph{individual constituent point, line, or shape} that, when combined with other constituent elements, makes up that mark. For example, although there are separate codes for one single straight line (26.17.08) and for multiple straight lines (26.17.01), there aren't separate codes for, say, three straight lines as opposed to four lines. So although four lines are more complex than three, neither the relevant codes, nor the total number of codes assigned, will necessarily capture that incremental complexity.\footnote{On the number of codes that the USPTO should properly assign to multiple lines, see \textit{supra} note~\ref{supra26}.} Likewise, because only ``\emph{significant} design elements'' are coded, any shape—no matter how detailed or intricate—will trigger only a single code (or multiple duplicative codes, if more than one code corresponds) unless some aspect of that shape is significant enough to warrant an additional design search code.\footnote{\textit{Design Search Code Guidelines}, \textit{supra} note~\ref{supra15} (guidelines 5, 6, 7, 10).}

For example, the marks shown below each have only one design search code, the design search code for cars (18.05.01), even though the mark on the left contains much more visual detail (e.g., more lines) than the mark on the right.

\begin{figure}[H]
\centering
\begin{subfigure}[h]{0.3\linewidth}
\includegraphics[width = \linewidth]{RN_5456370_drawing} \
\caption{Drawing submitted for Registration No. 5,456,370}
\end{subfigure}
\hspace{30pt}
\begin{subfigure}[h]{0.4\linewidth}
\includegraphics[width = \linewidth]{RN_3625662_drawing} \
\caption{Drawing submitted for Registration No. 3,625,662}
\end{subfigure}
\end{figure}
\par

\noindent And, perversely, both of these marks have fewer design search codes than the mark \includegraphics[scale = 0.2]{RN_7139454_drawing},\footnote{Registration No. 7139454.} which is coded with the design search code for a circle (26.01.21) in addition to the code for cars.

\item[c.] \textit{Inconsistency}. The USPTO's coding practices have been inconsistent in several respects. As noted, inconsistencies in coding quality due to human error can distort the data this study relies on. However, the USPTO's efforts to remedy hand-coding inconsistencies—and to improve the functionality of the codes in general—have introduced a different and perhaps greater distortion in the data.

Over the course of the time period examined here, the USPTO has made several changes to the set of design search codes themselves, although none of these changes has altered the overall structure of the codes.\footnote{These changes typically take one of three forms: (1) changing the visual elements to which certain codes correspond, (2) adding new codes for visual elements the USPTO has decided merit specific identification, and (3) removing visual elements from the scope of one code and placing them in another code's scope.} The USPTO's largest set of code changes occurred in January 2007. It introduced 75 new codes, most of which corresponded to a specific type of visual element that had previously fallen within the scope of a more general code (e.g., creating a code specific to \emph{more than one} star with four points (01.01.12), as opposed to only one such star, when previously any amount of four-point stars would receive the same code (01.01.02)). In addition to applying these changes prospectively to new applications, the USPTO applied the changes retrospectively to all then-``active'' applications.\footnote{\textit{Modifications to Design Search Codes}, \textit{supra} note [X].} But it didn't apply those changes to historical, inactive applications. So around 467,000 applications (53,423 of which are within the set of nonword trademark applications I examine here) were ineligible to be recoded, although the actual number of applications that were affected is likely much smaller.\footnote{NTD: To calculate this figure in the next draft.} Likewise, when the USPTO has subsequently introduced additional, incremental code changes, it seems to have applied those changes prospectively to new applications while only partially recoding old applications.\footnote{For example, in 2022, the USPTO removed the catchall code for ``heads of cats, dogs, wolves, foxes, bears, lions and tigers'' (03.01.16). In its place, the USPTO added three more specific codes for heads of felines (03.01.18), heads of dogs, wolves, and foxes (03.01.19), and heads of bears (03.01.20). The USPTO has applied this trio of new codes to subsequent applications. \textit{E.g.}, Serial No. 98,237,370. But it appears to have deleted all instances of the now-defunct catchall code without recoding the applications from which that now-defunct code was removed. \textit{E.g.}, Registration No. 1,133,264; Registration No. 6,432,880.}

The USPTO has also made procedural changes that have likely affected the consistency of coding quality over time. In late 2005, the USPTO began informing applicants of any design search codes assigned to their proposed marks and soliciting applicants' input on the assigned codes, even though the ultimate authority to determine the applicable codes rests with the USPTO.\footnote{\label{supra27} Notice of the Removal of the Paper Search Collection of Registered Marks That Include Design Elements from Trademark Search Library in Arlington, VA, 72 Fed. Reg. 35429, 35431 (June 28, 2007); Request for Comments on Removal of Paper Search Collection of Marks That Include Design Elements, \textit{supra} note~\ref{supra14}, at 36067.} Then, in mid-2007, the USPTO began soliciting the same input from all mark owners with active registrations that had visual elements.\footnote{In some cases, the USPTO did so only after first reviewing those active registrations and adopting self-sourced corrections to those registrations' codes, subject to any input from the mark owner. Coding of Design Marks in Registrations, \textit{supra} note~\ref{supra10}, at 81588.} Further, in 2008, the USPTO revised the trademark application regulations to require applicants to submit a description of any proposed mark that was not in plain text. The USPTO's logic was that being able to consult such descriptions would permit its coders to improve the accuracy of the design search codes assigned to proposed marks with visual elements.\footnote{Changes in the Requirement for a Description of the Mark in Trademark Applications, 73 Fed. Reg. 13780 (Mar. 14, 2008); Changes in the Requirement for a Description of the Mark in Trademark Applications, 72 Fed. Reg. 60609 (Oct. 25, 2007).}

These procedural changes are safeguards. In the abstract, they don't necessarily have to affect the codes assigned to applications. Indeed, if the USPTO coded with perfect accuracy and had applied the January 2007 code changes to all applications that were active at that time or later, then these procedural changes would not have any additional effect on the design search codes assigned to any application. However, we know the USPTO does not code with perfect accuracy. These procedural changes thus strongly suggest that the quality of the design search codes for any application that was active as of late 2005 or later will be better than the quality of the design search codes for any comparable earlier application that was inactive as of late 2005.

More generally, as discussed further in Section~\ref{sec:2}.\ref{subsec:2B}, these procedural changes point to potential endogeneity in my empirical analysis. As noted above, the reason the USPTO began dedicating increased resources to design search coding during 2005-2008 appears to have been suggestions that its design search coding \textit{often omitted relevant codes}. Thus, notwithstanding the USPTO's recoding of pre-2005 applications, it is possible that increases in the number of codes assigned to each mark since 2005 have resulted from an increase in granularity and attention to detail in the USPTO's coding practices, as opposed to changes in the visual elements of the coded marks themselves. This account does not straightforwardly explain the increase in applications with a high number of codes before 2005, seems somewhat implausible where a mark has an especially high number of codes (e.g., more than 15), and runs up against the robustness checks I include at the end of Section~\ref{sec:2}.\ref{subsec:2B}. However, this account points to the important reality that the USPTO is an evolving institution, and this study cannot fully account for diffuse or practically unobservable changes in USPTO practice over time.

\end{enumerate}

Given these limitations on the use of number of design codes as a metric, my statistical analysis below includes several other metrics. In particular, I also include information on the per-mark number of (a) different \emph{categories} of codes, (b) codes for visual elements that I term ``non-basic,'' and (c) codes corresponding to groups consisting of multiple visual elements.

\begin{enumerate}

\item[i.] \textit{Categories of codes}. Looking at the number of different \emph{categories} of codes a mark has—as opposed to the raw number of individual codes it has—is helpful because it eliminates some of the effects of duplicative coding discussed above. Specifically, where the duplicative codes are within the same category (e.g., the general code for cats and the specific code for heads of cats are both in category 03, the category for animals), counting at the category level eliminates the duplication.

This cuts both ways, though. If two or more separate visual elements in a mark simply \emph{happen} to fall within the same category, looking at the number of different categories coded will necessarily undercount the number of visual elements in the mark. For example, when coding the Moynat Réjane clasp mark referred to in the Introduction, \includegraphics[scale = 0.15]{RN_5139501_drawing},\footnote{Registration No. 5139501.} the USPTO assigned it codes for the circular shape of the clasp itself (26.01.06, 26.01.18, and 26.01.21), as well as for the overall rectangular shape of the mark (26.11.28). All of those codes are in a single category (category 26). Yet they correspond to two different visual elements in the mark. So, although focusing on the number of categories of codes per mark can sometimes provide a better measure of how many significant visual elements a mark has, it is not a panacea.

\item[ii.] \textit{Non-basic codes}. We can also look at which specific codes correspond to more complex shapes. In particular, there is almost no limit to how simple the shapes in category 26 (``geometric figures and solids'') can be. A single point\footnote{\textit{E.g.}, Registration No. 513,361.} or line\footnote{\textit{E.g.}, Serial No. 90,886,324.} would fall into this category. To a lesser extent, the same applies to certain divisions in category 01 (``celestial bodies, natural phenomena, geographical maps'')\footnote{For example, the design codes for a full moon (01.11.01) or a single drop (01.15.08) correspond to marks that come close to the simplicity of a single point. \textit{E.g.}, Registration No. 4,148,351.} and category 25 (``ornamental framework, surfaces or backgrounds with ornaments'').\footnote{For example, certain codes in category 25 could apply to marks that are effectively a simple square, \textit{e.g.}, Registration No. 4,149,753 (coded as 25.01.25, ``other framework or ornamental borders''), or a set of points, \textit{e.g.}, Registration No. 7,096,377 (coded as 25.03.05, ``backgrounds covered with dots (not mere stippling)'').} In contrast, there is a limit to how simple the shapes that correspond to ``human beings'' (category 02), ``animals'' (category 03), or ``natural phenomena'' (section 01.15) can be. Beyond a certain level of simplicity, such shapes would no longer be recognized as depicting anything at all. Rather, they would have fallen into the realm of codes corresponding to more abstract, basic shapes. So we might assume that, where one mark has more such ``non-basic'' codes than another mark, the first mark is likelier to be the more complex of the two—at least where the two marks have the same amount of codes. This may thus be a serviceable measure of visual complexity that doesn't rely on a mark's raw number of individual codes. However, even marks with zero non-basic codes may be quite complex. For example, the two marks below both have only one design search code: the design search code for ``backgrounds covered with circles or ellipses.''

\begin{figure}[H]
\centering
\begin{subfigure}[h]{0.4\linewidth}
\includegraphics[width = \linewidth]{RN_4159962_drawing} \
\caption{Drawing submitted for Registration No. 4,159,962}
\end{subfigure}
\hspace{30pt}
\begin{subfigure}[h]{0.275\linewidth}
\includegraphics[width = \linewidth]{RN_7096377_drawing} \
\caption{Drawing submitted for Registration No. 7,096,377}
\end{subfigure}
\end{figure}
\par


\noindent But the mark on the left is fairly intricate and contains a substantial amount of ellipses/circles. It may not fall into the category of basic shapes. Conversely, marks whose only design search codes are in categories other than 01, 25, or 26 can still be simple—for example, the mark \includegraphics[scale = 0.15]{RN_7099151_drawing},\footnote{Registration No. 7,099,151} whose only code is 06.01.04 (``mountains, mountain landscapes''). Accordingly, looking to the relative basicness of design search codes provides only a very rough measure of the complexity of the underlying marks.

\item[iii.] \textit{Multiple-element codes}. We can obtain another rough measure of complexity by examining codes that correspond to multiple, as opposed to single, iterations of a particular visual element. For example, design search codes 01.01.09 and 01.01.10 correspond to two stars and three or more stars, respectively, and design search codes 01.01.12, 01.01.13, and 01.01.14 correspond to more than one star with four points, five points, and six points, respectively.\footnote{Thus, single stars are only coded with the other codes in division 01 of category 01.} Being assigned such ``multiple-element'' codes relates to a mark's complexity, given that a greater number of elements (e.g., a greater number of stars) indicates greater complexity.\footnote{\textit{See supra} note~\ref{supra24}.} In particular, we might assume that, where one mark has more such multiple-element codes than another mark, the first mark is likelier to be the more complex of the two—at least where the two marks have the same amount of codes. However, multiple-element codes exist for only a few common shapes (e.g., stars, drops, trees). For all other shapes (e.g., flames, mountains, computers), the USPTO will assign the general code for that shape to both single and multiple iterations of that shape.\footnote{\textit{See Design Search Code Guidelines}, \textit{supra} note~\ref{supra15} (guideline 9) (``If two or more design elements fall within the same [section], the particular code is recorded only once.'').} For instance, a mark consisting of one cat\footnote{Registration No. 5,671,762.} and a mark consisting of multiple cats\footnote{Registration No. 4,033,028.} will both receive the same general code for cats (03.01.04), which does not differentiate as to quantity. Thus, although multiple-element codes do track complexity to some extent, the picture they provide is substantially incomplete.

\end{enumerate}

Indeed, although these alternative metrics do shed light on different facets of mark complexity, they are just as imperfect as the raw number of individual codes per mark. This is unavoidable. Design search codes are not intended to provide a comprehensive description of a mark's visual elements. Rather, they are intended to aid the USPTO and potential applicants in identifying preexisting marks that may conflict with a new proposed mark. Design search codes thus implicitly wager that some distinctions between marks (e.g., whether a mark consists of one or multiple cats) are not relevant enough to warrant separate codes, because being able to search using those distinctions would not materially improve one's ability to identify relevant preexisting marks.

Likewise, none of these metrics for complexity is immune to the issue of inconsistent coding. That said, neither the USPTO's 2007 changes to the codes nor its 2005-2008 procedural changes are particularly impactful when it comes to these metrics—with the important exception of multiple-element codes. The USPTO's 2007 changes to the codes generally took the form of adding new sections to preexisting divisions or making discrete amendments to sections.\footnote{Codes added in 2007 contained visual elements that would have previously corresponded to a more general code. For example, before the USPTO's 2007 changes, thought or speech clouds would have fallen into design search codes 01.15.06 (``clouds, fog''), 26.01.28 (``miscellaneous circular designs with an irregular circumference''), 26.09.28 (``miscellaneous designs with overall square shape''), 26.11.28 (``miscellaneous designs with overall rectangular shape''), or 26.15.28 (``miscellaneous designs with overall polygon shape''), depending on their specific shape. After the changes, thought or speech clouds received their own unique design search code (01.15.17), a new section within a preexisting division. Similarly, the USPTO also added a new \emph{division} (as opposed to a section) for Braille, Morse code, and sign language to category 28.} These changes did not affect the overall organization of code categories and divisions or the basic principle that each significant design element must receive an appropriate code. Thus, the circumstances under which these changes could have introduced inconsistency into the number of codes, number of codes in different categories, and number of non-basic codes per mark are limited.\footnote{Coding a mark using post-2007 design search codes, as opposed to pre-2007 codes, would only result in adding an additional code to that mark where, pre-2007, a general code (e.g., the code for clouds and fog) applied to two or more elements in a mark (e.g., a mark with both fog and a speech cloud) and, post-2007, a new code applied to only one of those elements (e.g., the new code for speech clouds). The USPTO's dataset doesn't allow us to calculate how often this happened, but, given the relatively small number of divisions affected by the changes, we can assume this would be an infrequent occurrence. [NTD: To calculate how often affected codes pair with new codes in post-2007 marks.] The same applies, a fortiori, to number of codes in different categories and number of ``non-basic'' codes per mark. Only one of the 2007 changes resulted in removing visual elements from one category and placing them in another, and none of those changes resulted in removing visual elements from a non-basic code to a basic code or vice versa.} The same goes for the smaller, incremental changes the USPTO has made to the codes since 2007.\footnote{NTD: To write a footnote on these changes in the next draft.} In contrast, because the 2007 changes added several multiple-element codes for common shapes, a considerable number of applications filed before 2007 might lack these codes when they should have them, meaning that multiple-element codes do not provide a helpful basis for analyzing marks in the years before 2007.\footnote{[NTD: To calculate how often multiple-element codes added in 2007 show up among post-2007 marks.]} Accordingly, the analysis that follows only examines the number of multiple-element codes per mark for marks filed after the USPTO implemented its 2007 code changes.

\subsection{Statistical Analysis}\label{subsec:2B}

\subsubsection{Descriptive Statistics}\label{subsubsec:2B1}

\addtocounter{figure}{-4}

This Section presents evidence that the number and proportion of applications for highly complex nonword marks filed at the USPTO have increased over the past few decades. To be clear, I am not claiming that nonword applications, in the aggregate, have become more complex.\footnote{This distinction is admittedly technical. As Figure~\ref{fig:5} suggests, the increase in nonword marks with a number of codes that is one or more standard deviations away from the mean number of codes is already pulling up the mean number of codes. Nonword marks are perhaps becoming more complex on average. However, by cautioning that I am not making claims about the complexity of nonword marks in general, I intend to highlight that my analysis focuses on a specific subset of nonword marks (those with a high number of codes), rather than nonword marks in general.} Rather, this is a story about outliers and near-outliers. Within the distribution of nonword applications filed at the USPTO, we are increasingly seeing applicants file for marks at the highly-complex end of the spectrum.

To understand this phenomenon, we first need to understand the baseline. Figure~\ref{fig:1} shows the total number of trademark applications—for both word and nonword marks—filed at the USPTO each year from 1986 through 2023. Also shown are the total number of those applications that were published for opposition, indicating that the USPTO determined they met all substantive statutory requirements for registration.\footnote{Figure~\ref{fig:1} is consistent with data previously reported by other scholars. \textit{See, e.g.}, Beebe \& Fromer, \textit{supra} note~\ref{supra4}, at 972 fig.2; Beebe, \textit{supra} note~\ref{supra8}, at 761 fig.1.} The number of USPTO-approved applications in the past two years is substantially lower because many applications filed in the past two years have not yet received a final decision on registrability from the USPTO.\footnote{The average time between application filing and registration or abandonment is currently over a year. \textit{Trademark Processing Wait Times}, USPTO, \url{https://www.uspto.gov/trademarks/application-timeline}.}

\begin{figure}[H]
\centering
\input{./graphics/figure_1.tex}
\caption{\label{fig:1} Total number of trademark applications (word and nonword) filed and approved at the USPTO per year (1986 - 2023)}
\end{figure}

As is evident, the total number of trademark applications filed at the USPTO per year has grown significantly, increasing roughly 800\% from 61,872 in 1986 to 542,900 in 2023. This increase has several drivers. Perhaps chief among them is the overall growth of the U.S. economy.\footnote{\textit{See} Beebe, \textit{supra} note~\ref{supra8}, at 761.} However, more recently, the ongoing increase in filings has partly resulted from Amazon's requirement that businesses who wish to be listed on the Amazon Brand Registry have at least a trademark application for their brand/product name. Among other things, this has generated a deluge of trademark applications for ``nonsense'' marks adopted solely for the purpose of inclusion on the Brand Registry (e.g., ``ELXXROONM, SUJIOWJNP, XUFFBV, and LXCJZY'') and has also resulted in a number of fraudulent applications.\footnote{Jeanne C. Fromer \& Mark P. McKenna, \textit{Amazon's Quiet Overhaul of the Trademark System}, 118 \textsc{Calif. L. Rev.} (forthcoming 2025), \url{https://papers.ssrn.com/sol3/papers.cfm?abstract_id=4870984}.} Moreover, in recent years, subsidies provided by regional Chinese governments to Chinese citizens who secure U.S. trademark registrations may also have incentivized applications, including, again, a number of fraudulent applications.\footnote{Beebe \& Fromer (Fake Trademark Specimens); Fromer \& McKenna.} Likewise, the recent, widespread practice of individuals filing weak, opportunistic applications to register words/phrases that become viral or widely used (e.g., ``LET HIM COOK''\footnote{Serial No. 98,223,135.}, ``WOKE MIND VIRUS''\footnote{Serial no. 98,670,319}) may have pushed overall filing numbers up.\footnote{Cuatrecasas, \textit{supra} note [X], at 1355–56.}

Applications for nonword marks exhibit a similar pattern. Figure~\ref{fig:2} shows the number of applications for nonword marks for each year from 1986 through 2023, as well as the total number of those applications that received approval. The rate of increase in numbers of applications here is lower than it is for all marks: applications for nonword marks have only increased from 2,562 in 1986 to 15,189 in 2023 (about a 500\% increase). Indeed, as Figure~\ref{fig:3} shows, applications for nonword marks have \emph{decreased} as a percentage of total applications. This phenomenon is probably best understood as a result of the recent drivers of increased trademark filings overall. The application-incentivizing effects of Amazon's Brand Registry, as well as the general tendency toward seeking applications for viral or widely used words/phrases, apply almost exclusively to word marks (including marks that blend words and visual elements). On a separate note, the proportion of nonword applications that receive approval is generally higher than it is for all marks (the average percentage of nonword marks approved from 1986 through 2021 is 81\%, as compared to 73\% for all marks during the same period).\footnote{The approval rate during the same time period for marks the USPTO has categorized as consisting of words (whether or not in combination with visual elements) is also 73\%.} That is, proposed nonword marks have a better chance of securing approval than proposed marks in general and, by extension, than word marks. Section~\ref{sec:2}.\ref{subsec:2B}.\ref{subsubsec:2B2} returns to this point.

\begin{figure}[H]
\centering
\input{./graphics/figure_2.tex}
\caption{\label{fig:2} Total number of nonword trademark applications filed and approved at the USPTO per year (1986 - 2023)}
\end{figure}

\begin{figure}[H]
\centering
\input{./graphics/figure_3.tex}
\caption{\label{fig:3} Nonword trademark applications as a percentage of total applications filed at the USPTO per year (1986 - 2023)}
\end{figure}

Design search codes fit into this picture in a somewhat counterintuitive way. All marks with visual elements receive at least one design search code. The average number of design search codes per nonword mark from 1986 through 2023 is 3.18 (sd = 2.12), and the median number of codes is 3. As that pair of statistics suggests, a mark's number of codes can be much higher than the median, leading to a highly right-skewed distribution of codes per application. Indeed, the highest number of design search codes for any nonword mark from 1986 through 2023 is 67, and about 12\% of all nonword marks during that period have more than five design search codes (i.e., are more than one standard deviation to the right of the mean). Going forward, I'll refer to applications for marks to which the USPTO assigns more than five design search codes as ``high-code'' applications. For the reasons explained in Section~\ref{sec:2}.\ref{subsec:2A}, these marks tend to exhibit considerable visual complexity, with some exhibiting extreme complexity.

%\begin{figure}[H]
%\centering
%\input{./graphics/figure_4.tex}
%\caption{\label{fig:4} Frequency distribution of number of design search codes per nonword mark (1986 - 2023)}
%\end{figure}

Figure~\ref{fig:4} shows the mean and median number of codes per mark per year from 1986 through 2023. Over time, the mean number of codes has increased, edging upward from 3.20 codes per nonword mark in 1986 to 3.41 codes per mark in 2023. The median is 3 in all years except for between 1994 and 2001, when it drops to 2. These statistics are consistent with an increasing number of outlier applications with an especially high number of codes.

\begin{figure}[H]
\centering
\input{./graphics/figure_4.tex}
\caption{\label{fig:4} Mean and median number of design search codes for nonword trademark applications filed at the USPTO per year (1986 - 2023)}
\end{figure}

Zooming in slightly more, Figure~\ref{fig:5} shows the yearly numbers of nonword applications filed at the USPTO with more than 5, 10, and 15 design search codes. Each shows a sharp increase in recent decades. More impressively, nonword applications with more than 10 and 15 codes exhibit a clear increase even when shown as percentages of the total number of all applications per year. (The trend for nonword applications with more than five codes is more ambiguous—visual inspection reveals an increasing trend, but the proportion of such applications in 1986 is marginally higher than in 2023.) Put otherwise, applications with the highest numbers of codes have increased as a proportion of all applications, even though the larger set to which they belong (nonword applications) is simultaneously becoming a smaller proportion of all applications.\footnote{[Notably, we observe the same trend even if we narrow the data down to include only applications that ultimately registered or only applications whose design search codes would have been subject to the USPTO's above-discussed review-and-recoding process.]}

\begin{figure}[H]
\centering
\input{./graphics/figure_5.tex}
\caption{\label{fig:5} Number of nonword applications filed at the USPTO with more than 5, 10, and 15 design search codes per year (1986 - 2023)}
\end{figure}

\begin{figure}[H]
\centering
\input{./graphics/figure_6.tex}
\caption{\label{fig:6} Percentage of total applications that consists of nonword applications with more than 5, 10, and 15 design search codes per year (1986 - 2023)}
\end{figure}

Thus, in recent decades, high-code applications have become more common. This remains true even when we account for the broader increase in applications. However, as noted, a mark's number of codes doesn't perfectly correspond to its complexity. The other design search code–based metrics discussed in Section~\ref{sec:2}.\ref{subsec:2A} can shed additional light on the underlying pattern revealed in the above data:

\begin{enumerate}

\item[i.] \textit{Categories of codes}. For the period from 1986 to 2023, the mean number of categories per nonword application is 1.85 (sd = 1.03), and the median number of categories is 2. Naturally, the range of values here is narrower than it is for number of codes: the nonword application with the most categories has 20. The change over time in the mean and median number of categories per year mirrors the change in the mean and median number of codes per year: the mean increases slightly, from 1.84 in 1986 to 1.87 in 2023\footnote{The mean number of categories reaches a minimum of 1.63 in 2001 and a maximum of 2.01 in 2009.}, and the median stays constant at 2, except for between 1995 and 2001, when it drops to 1. Moreover, as with number of codes, applications with high numbers of different categories have grown over time, and applications with especially high numbers of different categories have grown as a percentage of all applications.\footnote{Nonword applications with more than 5 categories have risen as a percentage of all applications by about 0.1\% from 1986 to 2023.} All in all, this data confirms what we already know: we are seeing greater numbers of nonword applications that have many design search codes, and, concomitantly, a greater likelihood of having different categories of codes.\footnote{Whereas the mean number of codes for all marks is 3.18, the mean number of codes for marks with more than 3 different categories is 6.97, and the mean number of codes for marks with more than 5 different categories is 10.87.}

\item[ii.] \textit{Non-basic codes}. Non-basic codes reveal a different side to this story. From 1986 to 2023, the number of applications with at least one non-basic code has markedly risen. But it has fallen as a percentage of the total number of applications per year. And although the number of applications with especially high numbers of non-basic codes has increased over that time period, that may say more about the general trend toward higher numbers of marks with many design codes than it says about a trend toward higher numbers of ``non-basic'' shapes. Indeed, the average number of basic codes per application over the same time period shows a slight upward trend. And if we look at applications with particularly high numbers of basic codes, we see an increase \emph{even if we take into account the broader increase in number of applications filed per year}. Put otherwise, it seems we are seeing a trend toward marks with high numbers of codes that correspond to basic, simple shapes.

\item[iii.] \textit{Multiple-element codes}. [Examining the prevalence of multiple-element codes in post-2007 applications yields inconclusive results. From early 2007 to 2023, the mean number of multiple-element codes per mark does not show a clear trend. Nor does the percentage of applications per year with one or more multiple-element codes. If anything, these metrics show an increase during that time period, but the trend is a very slight one.\footnote{The year 2010 seems to have seen an especially high proportion of these codes being applied to applications. Perhaps this is due to chance. Alternatively, one might conjecture that it is because, in 2010, the USPTO's 2007 code changes were still relatively recent, and examiners may have initially overapplied the new design search codes.} This is not particularly surprising. Close to half of the multiple-element codes are also basic codes. So, holding everything else constant, if we are seeing an increasing number of marks with high numbers of basic codes, we might expect to see a general increase in the number of multiple-element codes in the population of applications.]

\end{enumerate}

The upshot here is somewhat counterintuitive. We are seeing more and more applications for marks with a high number of codes, but the codes themselves may not correspond to visual elements that are especially complex. Instead, the codes for high-code applications may often—though not always—correspond to basic, simple, or abstract visual elements (e.g., lines or circles), as opposed to more representational, detailed elements (e.g., giraffes or space rockets). That's not particularly surprising. Basic codes are highly utilized. About [44\%] of the codes assigned to nonword applications from 1986 through 2023 are basic, even though basic codes constitute only [14\%] of all unique codes that could be assigned to marks. But high-code applications are also, on average, likelier to have more basic codes. The average percentage of basic codes per high-code application is [47\%], even though the average percentage of basic codes per mark for all applications is [40\%]. At the same time, basic codes are not the \emph{sole} driver of the increase in high-code marks. As shown in the robustness checks at the end of this Section, that increase remains observable even if we control for the number of basic codes per mark.

The way this cashes out, in practical terms, is by giving us two distinct pictures of complexity. On the one hand, applications with very high numbers of codes are often detailed two-dimensional designs. Some may be more representational and others more abstract, but they are basically still traditional logos: two-dimensional designs that are placed on products or displays associated with products. Take, for example, the following elaborate image, registered by Radyne Corporation, an induction heating manufacturer, and used on its promotional materials:

\begin{figure}[H]
\centering
\begin{subfigure}[h]{0.3\linewidth}
\includegraphics[width = \linewidth]{RN_4392083_drawing} \
\caption{Drawing of mark submitted to USPTO}
\end{subfigure}
\hspace{30pt}
\begin{subfigure}[h]{0.425\linewidth}
\includegraphics[width = \linewidth]{RN_4392083_specimen} \
\caption{Specimen of mark use}
\end{subfigure}
\caption*{Registration No. 4,392,083}
\end{figure}
\par

\noindent Or consider the following detailed scene, registered by A.H. Riise, a Danish rum manufacturer, and used on the packaging for its liquor:

\begin{figure}[H]
\centering
\begin{subfigure}[h]{0.4\linewidth}
\includegraphics[width = \linewidth]{RN_6115490_drawing} \
\caption{Drawing of mark}
\end{subfigure}
\hspace{30pt}
\begin{subfigure}[h]{0.25\linewidth}
\includegraphics[width = \linewidth]{RN_6115490_specimen} \
\caption{Specimen of mark use from A.H. Riise's website}
\end{subfigure}
\caption*{Registration No. 6,115,490}
\end{figure}
\par

\noindent On the other hand, many applications with very high numbers of codes are less traditional. They are sophisticated forms of trade dress, consisting of highly specific shapes or patterns that constitute a product or some component thereof. Often they are three-dimensional, and their overall visual impact may be subtler, or more conceptual, than that of the traditional logos above. For example, Amazon has received USPTO approval for a trademark in the appearance of the Amazon ``Spheres,'' a set of large domed conservatories built in a vaguely neofuturist architectural style and located at Amazon's Seattle campus:

\begin{figure}[H]
\centering
\includegraphics[scale = 0.45]{SN_90772818_drawing} \
\caption*{Drawing submitted for Registration No. 90,772,818}
\end{figure}
\par

\noindent Likewise, consider Dior's recent application for the shape of a handbag it sells: \

\begin{figure}[H]
\centering
\includegraphics[scale = 0.45]{SN_98174172_drawing} \
\caption*{Drawing submitted for Registration No. 98,174,172}
\end{figure}
\par

\noindent And, in a similar vein, Levi's has registered the shape of the entire back pocket of the jeans it sells:

\begin{figure}[H]
\centering
\begin{subfigure}[h]{0.25\linewidth}
\includegraphics[width = \linewidth]{RN_2791156_drawing} \
\caption{Drawing of mark}
\end{subfigure}
\hspace{30pt}
\begin{subfigure}[h]{0.25\linewidth}
\includegraphics[width = \linewidth]{RN_2791156_specimen} \
\caption{Specimen of mark use}
\end{subfigure}
\caption*{Registration No. 2,791,156}
\end{figure}
\par

In other words, nonword applications with very high numbers of codes generally fall into one of two buckets: complex logos or highly specific trade dress. For applications in the former bucket, the large number of distinct, separately identifiable visual elements they contain is what accounts for the numerosity of their codes. For applications in the latter bucket, there's a bit more nuance. The numerosity of these applications' codes flows more from their attempt to capture a large amount of specific details, many of which, if considered individually, may not seem particularly distinct. Thus, the back pocket of a pair of Levi's jeans may not seem particularly ``complex'' to a casual observer, but 15 codes (e.g., ``polygons touching or intersecting'' (26.15.16),  ``polygons inside another polygon'' (26.15.20)) are needed to account for the pocket's stitching, shape, and overall composition.

The subset of codes I've termed ``basic'' is again helpful in understanding these two buckets. Radyne's mark has zero basic codes, and only two of the 18 codes for A.H. Riise's mark are basic. In contrast, 15 of the 25 codes for the Amazon mark above, 23 of the 24 codes for the Dior mark above, and 13 of the 15 codes for the Levi's mark above are basic. This makes sense. Marks like Radyne's and A.H. Riise's involve many representational shapes (e.g., people, animals, trees). The Amazon, Dior, and Levi's applications involve representational shapes to some extent, but they largely focus on the structural details of objects (e.g., the contours of a building or a handbag). Those structural details are likelier to correspond to relatively simple, geometric shapes than they are to correspond to figurative, representational shapes. So naturally those applications will correspond to a greater number of basic codes. 

These are generalizations, of course. Some applications buck this trend: they are for complex logos that are composed of geometric shapes, or they are for a product shape that includes human or animal figures. But, overall, we are seeing more businesses apply for, on the one hand, designs that are effectively logos with a high number of visual elements and, on the other, designs that are highly specific and detailed agglomerations of abstract shapes associated with product features, many of which reach beyond the realm of traditional logos. The balance of this Part seeks to answer why.

\subsubsection{Inferential Statistics}\label{subsubsec:2B2}

The increase in visually complex applications may not be surprising. Previous empirical studies on trademark prosecution have shown that applications for ``nontraditional'' marks—for instance, marks consisting only of nonvisual elements like sound or scent—have increased in recent decades, both at the USPTO and at some of its counterparts in other countries.\footnote{\textit{See} Carolina Castaldi, \textit{The Economics and Management of Non-Traditional Trademarks}, \textit{in} \textsc{The Protection of Non-Traditional Trademarks} 257, 265 fig.13.1; \textit{Sound Marks}, \textit{supra} note~\ref{supra23}, at 2397 fig.I; \textit{see also} Mitchell Adams \& Amanda Scardamaglia, \textit{Non-Traditional Trademarks: An Empirical Study}, \textit{in} \textsc{The Protection of Non-Traditional Trademarks} 37, 45 fig.2.1.} As for visual marks in particular, Deborah R. Gerhardt and Jon McClanahan Lee have shown that nonword applications—and, specifically, applications for marks consisting of color alone—have increased at the USPTO.\footnote{\textit{Owning Colors}, \textit{supra} note~\ref{supra23}, at 2539–40 figs.XI, XII, XIII.} As those authors suggest, this increase may in part be due to the U.S. Supreme Court's 1995 decision in \textit{Qualitex Co. v. Jacobson Products Co.}, which held that a single color could be a trademark, as long as it had acquired distinctiveness.\footnote{\textit{See id.}.} Similarly, Irene Calboli has observed in qualitative terms that nontraditional marks consisting of highly specific trade dress—e.g., the shape of a handbag—are increasingly common.\footnote{Irene Calboli \& Martin Senftleben, Introduction, \textit{in} \textsc{The Protection of Non-Traditional Trademarks} 1; Irene Calboli, \textit{Chocolate, Fashion, Toys and Cabs: The Misunderstood Distinctiveness of Non-Traditional Trademarks}, 49 \textsc{Int'l Rev of Intell. Prop. \& Competition L.} 1 (2018).}

Against that background, one might interpret higher numbers of high-code applications as merely one facet of a broader uptick in filings for nontraditional marks, whether they consist of visual elements, sound, scent, or something else. But that's not exactly right. As previously noted, although the number of nonword applications filed per year has increased in recent decades, the share of those applications has \emph{decreased} if we account for the broader increase in applications for all marks, including word marks. In contrast, the share of high-code applications has increased, even accounting for the broader increase in all applications. That suggests that visually complex nonword marks are following a different trajectory from nonword marks in general. And if that's the case, then what accounts for that trajectory?

One possible answer is that businesses are increasingly filing for complex nonword marks because greater complexity increases their chances of securing a registration. At face value, that seems eminently plausible. Recall Section \ref{sec:1}.\ref{subsec:1B}'s discussion of the \textit{Seabrook} test for the inherent distinctiveness of trade dress. As the USPTO and courts have indicated, that test in substance deems a mark inherently distinctive if it is unique or uncommon in its commercial context.\footnote{See \textit{supra} note [X].} For example, proposing a simple, unadorned green circle on a coffee cup (\includegraphics[scale = 0.2]{RN_5754170_drawing})\footnote{Registration No. 5754170.} as a trademark for coffee drinks is not likely to pass the \textit{Seabrook} test, because circles of any color are common shapes in all fields.\footnote{\textit{In re} Starbucks Corp., Nos. 85792872 \& 86689423, 2019 TTAB LEXIS 8 (T.T.A.B. July 17, 2018) [non-precedential].} In contrast, using the Starbucks logo (\includegraphics[scale = 0.125]{RN_4538053_drawing}) for coffee drinks easily passes the \textit{Seabrook} test, because a circle containing a siren wearing a crown adorned with a star is not necessarily common in that field. Intuitively, the more visual elements one adds to a mark, the less likely it is that all of the elements in the mark, taken as a whole, will be commonplace in the mark's commercial context. So, theoretically, businesses could have an incentive to lard their applications with additional visual elements to better their chance of registering their mark.

Conventional wisdom may also support that account. When a client seeks to register a word mark that risks a rejection by the USPTO for being nondistinctive, trademark lawyers routinely suggest that the client combine the work mark with a nonverbal, visual element in order to reduce that risk.\footnote{[Cites.] \url{https://www.cll.com/OnMyMindBlog/adding-a-word-or-distinctive-design-to-a-mark-sometimes-can-avoid-a-likelihood-of-confusion}; \url{https://smithhopen.com/2020/12/01/the-right-to-your-name/}; cf. Fromer \& McKenna. Lawyers who make this suggestion will also generally note that the client should be open to including a disclaimer of the mark's verbal element, if the USPTO expresses the view that the matter should be disclaimed.} The logic here is that adding a visual element to a word mark subjects the mark to the \textit{Seabrook} analysis, because the mark is now not only a string of characters but a design as well. And as long as that design is not exceedingly common, \textit{Seabrook} is a relatively permissive test. This logic is only one step away from a strategy of adding more visual elements to a nonword mark (as opposed to a word mark) in order to increase its chances of receiving USPTO approval.

Indeed, the data suggest such a strategy could be effective. Fitting a regression model on the USPTO data discussed above suggests that having more design search codes improves an applicant's odds of receiving USPTO approval of their mark. To run that regression, I first exclude any applications filed in 2022 or later, because, as noted, there is a nontrivial number of applications filed in the last couple of years on which the USPTO has not yet reached a final decision. Because the dependent variable—USPTO approval—is binary, I use logistic regression with number of design search codes as the independent variable. Consistent with similar models in the literature, I include as dummy explanatory variables (1) the class of goods/services for which the mark was filed, (2) the year in which the application was filed, and (3) whether the application was filed pro se as opposed to by an attorney. Table~\ref{table:1} shows the intercept and coefficient of this regression model, omitting the dummy explanatory variables for (1) through (3) above.\footnote{W. Michael Schuster, Miriam Marcowitz-Bitton \& Deborah R. Gerhardt, \textit{An Empirical Study of Gender and Race in Trademark Prosecution}, 94 \textsc{S. Cal. L. Rev.} 1407, 1451 (2021).}

\input{table_1.tex} \par

[NTD: To run this regression with the somewhat odd maximum-class-only dummy variables used in the Schuster et al. piece—effect of n is probably very similar.] \\

\noindent In the table above, the ``estimate'' value of n indicates that, holding everything else constant, each additional design search code in a nonword trademark application increases an applicant's odds of USPTO approval by about $e^{0.0905}-1 \approx 10\%$.\footnote{[NTD: Because no nonword mark can have zero design search codes, this model effectively gives all marks a 10\% chance of securing approval by virtue of merely existing. To consider this point in the next draft.]} This increase is highly statistically significant (with a p-value of 0.000), although the absolute percentage change ($\sim10\%$) is somewhat modest. However, that percentage change becomes quite substantial where an application has more than five codes ($\ge 72\%$) and even more so where an application has more than ten codes ($\ge 170\%$). In other words, having a high number of codes can significantly better an applicant's odds of receiving USPTO approval. Although an approved mark doesn't automatically register (there are additional procedural requirements before it can do so), receiving approval means that the USPTO has determined that the mark itself meets all substantive statutory requirements for registration. In any case, number of design search codes is also a statistically significant predictor of whether a mark ultimately registers—each additional design search code increases odds of registration by about 4\%. Given these results, it at least plausible that the recent increase in marks with high numbers of design codes represents a strategy by applicants to increase their odds of approval and registration. 

But this strategy doesn't explain everything. \textit{Seabrook} was decided in 1977, about a decade before the beginning of the time period examined in my empirical analysis. It doesn't take much trial-and-error to figure out that (1) \textit{Seabrook} effectively says that unusual nonword marks are likelier to be registered, and (2) one way to make a mark unusual is to add more things to that mark. Thus, one might expect that a strategy of packing visual elements into a mark would take off as soon as a few applicants realized it worked. And that strategy would then remain popular—that is, applicants would employ that strategy at roughly consistent rates—as long as the law remains the same.\footnote{Somewhat analogously, consider that there were pronounced increases (1) in applications for color marks after the Supreme Court's decision in \textit{Qualitex}, which held that colors could be marks upon a showing of secondary meaning, and (2) in LGBTQ+-oriented applications in the years immediately following the Supreme Court's decision in \textit{Matal v. Tam}, which struck down the Lanham Act's prior prohibition on the registration of ``disparag[ing]'' marks. \textit{Owning Colors}, \textit{supra} note~\ref{surpa23}, at [X]; Michael Goodyear, \textit{Queer Trademarks}, 2024 \textsc{U. Ill. L. Rev.} 163.} But that's not what we see at all. Instead, Figure~\ref{fig:7} shows that marks with the highest number of design search codes seem, first, to moderately \textit{decrease} as a percentage of total applications during the late '80s and early '90s, after which they pick up substantially. Although it's possible that this is the natural pattern by which applicants adopted a visual element–packing strategy, it seems likelier that other factors are at play.

Context strongly suggests one such factor is the dramatic increase in total numbers of trademark filings in recent decade. As discussed, applications for word marks have driven that increase. But as applicants file for word marks at increasingly elevated levels, they are also filing for highly complex nonword marks at increasingly elevated levels. Figure~\ref{fig:8} gives an impressionistic sense of this trend by superimposing the yearly proportion of nonword marks with greater than 10 or greater than 15 codes over the yearly number of trademark applications.

\addtocounter{figure}{-3}

\begin{figure}[H]
\centering
\input{./graphics/figure_8.tex}
\caption{\label{fig:8} Total applications filed at the USPTO (\textbf{\color{oxfordblue}left axes}) and share of those applications that are nonword with $>10$ and $>15$ codes (\textbf{\color{seafoam}right axes}) (1986 - 2023)}
\end{figure}

As the next Section discusses further, one compelling way to understand this pattern is as the result of a fundamental shift in the branding environment—and, indeed, in the broader aesthetic environment. In particular, qualitative accounts of branding trends in recent years suggest that, as ever-growing numbers of word marks have saturated trademarks' linguistic territory, business have come to focus more on nonverbal branding elements as a way to distinguish themselves. In other words, as less and less of the domain of words is available to be claimed as a mark, businesses' focus on creating unique, memorable trademarks may intensifying within the domain of the visual, leading to a greater emphasis on claiming specific images, shapes, and patterns as marks.

It is not straightforward, however, to test this causal thesis. The yearly numbers of nonword trademarks, high-code marks, and all trademarks in general have each increased over time. The correspondence among them reflects a general increase in filings more than it reflects the effect of the increase in one type of filing on another type of filing.

[To come: writeup of increase in numbers of applications for high-code marks in international classes where depletion of wordmarks is severe; writeup of robustness checks.]

\subsection{The New Trademark Complexity}\label{subsec:2C}

In this context, we could view rising visual complexity as the flipside of a parallel phenomenon in the world of word marks: the recent increase in linguistically complex marks. Although this Article is the first to examine rising visual complexity in design search codes, rising linguistic complexity is already documented in the literature. And tellingly, just like visual complexity, linguistic complexity is linked to the growing number of trademarks overall.

Through an analysis of, among other data, the same USPTO dataset I examine here, Barton Beebe and Jeanne Fromer have shown that a high proportion of words that would be desirable trademarks—namely, frequently-used English words, single-syllable words, and common surnames—are either already registered as marks or are confusingly similar to an already-registered mark.\footnote{Beebe \& Fromer, \textit{supra} note~\ref{supra4}, at 981.} This ``depletion'' of desirable marks has had two key effects. First, businesses are increasingly using similar marks where they can do so without creating confusion (e.g., one business uses YALE for higher education, another business uses YALE for door locks), a phenomenon Beebe and Fromer term trademark ``congestion.''\footnote{Beebe \& Fromer, \textit{supra} note~\ref{supra4}, at 1012–21; Gerhardt \& McClanahan, \textit{supra} note [X], at 586, 586 n.9.} Second—and more importantly for the present analysis—businesses are increasingly seeking to register trademarks that have more characters, syllables, and words in them.\footnote{Beebe \& Fromer, \textit{supra} note~\ref{supra4}, at 999–1003.} That is, the scarcity of desirable marks is pushing businesses toward increased ``trademark length, complexity, and bulkiness.''\footnote{\textit{Id.} at 999–1024.}  In particular, Beebe and Fromer show that, from 1985 through 2016, the average word count per mark has increased by about [0.3], the average syllable count per mark has increased by about [0.5], and the average character count per mark has increased by about [1.5].\footnote{[NTD: To confirm these figures with Barton.] \textit{Id. 1002}}

Anecdotal evidence backs these figures up. Beebe and Fromer cite various journalistic accounts of industries in which trademark depletion has encouraged market players to adopt long, awkward names.\footnote{\textit{Id.} at 949, 949 n.6 (citing Sam Richards, \textit{FKA Twigs, Slaves, Deers: Are We Running Out of Band Names?}, \textsc{Chi. Trib.} (Jan. 7, 2015), \url{https://www.theguardian.com/music/2015/feb/17/fka-twigs-slaves-deers-band-name-shortage}; Kathleen Lee-Joe, \textit{The Beauty Industry Has Run Out of Makeup Names}, \textsc{DailyLife} (Jan. 16, 2015), \url{http://www.dailylife.com.au/dl-beauty/makeup/the-beauty-industry-has-run-out-of-makeup-names-20150115-12qw2i.html}).} I have in other work discussed fashion law commentator Julie Zerbo's suggestion that fashion brands may be adopting this strategy to some extent.\footnote{Cuatrecasas, \textit{supra} note~\ref{supra13}, at 1359 (citing \textit{A Slew of New Brand Names Raises the Question: Is Fashion Running Out of Trademarks?}, \textsc{Fashion L.} (Oct. 24, 2019), https://www.thefashionlaw.com/are-fashionbrands-running-out-of-trademarks).} Nor does it take much effort to find additional examples of businesses and consumers discussing the increasing unwieldiness of brand names in a given industry.\footnote{\textit{E.g.}, Joey Rippole, \textit{What Happens When an Industry Runs Out of Names?}, \textsc{Branding Mag} (Sept. 7, 2020), \url{https://www.brandingmag.com/2020/09/07/what-happens-when-an-industry-runs-out-of-names/}.} Indeed, trademark lawyers and brand professionals may have contributed to this trend to some extent: in an environment where it is de rigueur for trademark lawyers and other brand professionals to identify one or several existing marks that conflict with a client's proposed mark, it's standard to advise clients to consider \textit{adding} another distinctive term to the proposed mark.\footnote{\textit{E.g.}, Reema Pangarkar, \textit{Adding a Word or Distinctive Design to a Mark Sometimes Can Avoid a Likelihood of Confusion}, \textsc{Cowan, Liebowitz \& Latman} (Jan. 10, 2024), \url{https://www.cll.com/OnMyMindBlog/adding-a-word-or-distinctive-design-to-a-mark-sometimes-can-avoid-a-likelihood-of-confusion}.}

But increasing linguistic complexity is not a new concept, though it may now be verifiably on the rise. For almost a century, if not more, the specter of unwieldy brand names has loomed. In 1935, Felix S. Cohen observed that ``if courts prevent a man from exploiting certain forms of language which another has already begun to exploit, the second user will be at the economic disadvantage of having to pay the first user for the privilege of using similar language \textit{or else of having to use less appealing language (generally) in presenting [their] commodities to the public}.''\footnote{Felix S. Cohen, \textit{Transcendental Nonsense and the Functional Approach}, 35 \textsc{Colum. L. Rev.} 809, 816 (1935) (emphasis added).}  And already in 1957 the \textit{New York Times} reported that ``every day between three and five hundred new products make their appearance. The sheer number of new things requiring zippy, colorful identification is appalling to industrialists, who live in frankly avowed dread of the day when the demand for names will, once and for all, exceed the supply.''\footnote{Edith Efron, \textit{Brand New Brand Names}, \textsc{N.Y. Times}, July 7, 1957, at 140.} In 1975, the \textit{Times} was even more direct, running the sidebar headline ``Product Names Get Longer and Wordier.''\footnote{Philip H. Dougherty, \textit{Advertising: How Children Learn: Print vs. TV}, \textsc{N.Y. Times}, Dec. 13, 1976, at 60.} The story behind the brand name EXXON—born in the twentieth century and long cited as a prototypical example of a fanciful mark\footnote{\textit{E.g.}, \textsc{McCarthy}, \textit{supra} note~\ref{supra5}, \S 11:8; William M. Landes \& Richard A. Posner, \textit{Trademark Law: An Economic Perspective}, 30 \textsc{J.L. \& Econ.} 265, 273 (1987).}—is a case in point. When the Standard Oil Company changed its name to Exxon Corporation in 1973, it did so partly due to its inability to use the shorter mark ESSO, given conflicts with preexisting marks. But not even EXXON was free of conflicts. For its new name, Exxon had to pay then-Nebraska governor J. James Exon, owner of ``Exon's Inc.,'' to change his business's name to ``J. J. Exon Co.'' In effect, Exxon shifted the burden of linguistic complexity to Exon, who accepted the addition of more letters to his mark in exchange for cash.\footnote{James C. Tanner, \textit{Name Change Brings Exxcedrin Headaches and Costs Approxximately \$100 Million}, \textsc{N.Y. Times}, Jan. 9, 1973, at 44.}

Visual complexity is different. To be sure, businesses have long focused on building consumer recognition of images associated with their brands, such as logos and the like.\footnote{\textit{E.g.}, Jack R. Ryan, \textit{Company Symbol Bears Heavy Load}, \textsc{N.Y. Times}, May 10, 1959, at F1 (``Major industrial concerns are spending millions developing and promoting new corporate symbols—graphic trade-marks they hope will become famous and stamp their company identifies unforgetably [sic] in the public mind.'').} And some of the oldest corporate images and symbols (e.g., Levi's ``two horses'' logo, \includegraphics[scale = 0.225]{RN_523665_drawing}, registered in 1950 and used since the nineteenth century)\footnote{Registration No. 523665 (8 design codes); Mads Jakobsen \& David Shuck, \textit{Vintage Levi's 501 Jeans—The Ultimate Collector’s Guide}, \textsc{Heddels} (Apr. 16, 2024), \url{https://www.heddels.com/2016/11/vintage-levis-501-jeans-the-ultimate-collectors-guide/}.} are quite complex—more than many current logos and branding elements.\footnote{\label{supra17} \textit{E.g.}, Ben Schott, \textit{Debranding Is the New Branding}, \textsc{Bloomberg} (Mar. 7, 2021, 8:00 AM), \url{https://www.bloomberg.com/opinion/articles/2021-03-07/debranding-is-the-new-branding-for-burger-king-warner-bros} (discussing a trend among major brands toward revising their historical branding to be more minimalist). \textit{But see} Chris Kelly, \textit{Why Marketers Keep Refreshing Brands Instead of Betting on Splashy Ads}, \textsc{Marketing Dive} (July 10, 2024), \url{https://www.marketingdive.com/news/companies-rebranding-in-2024/719804/} (``For almost a decade, traditional brands looked at tech companies for design cues, pushing their designs toward a minimalism that erased identity and equity. That trend could be on the decline . . . .'').} But much of the visual complexity responsible for the rise in high-code applications seems to be a relatively new phenomenon. Looking at the entirety of the USPTO's data, the fifty nonword trademark applications with the most design search codes were all filed in 2006 or later. Perhaps more stunningly, of the fifty nonword trademark \textit{registrations} with the most design search codes, 22 of them (44\%) were registered in the past five years. Those 22 marks neatly split into the two buckets of high-code marks discussed in Section~\ref{sec:2}.\ref{subsec:2B}. About a fourth of them are registered by the Marc Anthony Group for different arrangements of branding elements on cans of White Claw Hard Selzer (e.g., \includegraphics[scale = 0.25]{RN_6774781_drawing}),\footnote{Registration No. 6774781.} a popular alcoholic drink. About a fifth of them are registered by Korea Ginseng Corporation for the specific images (e.g.,  \includegraphics[scale = 0.15]{RN_6263939_drawing})\footnote{Registration No. 6263939.} it incorporates into the packaging of the ginseng products it sells internationally. The other dozen or so marks are detailed two-dimensional images registered by various individuals and entities for smaller-scale (and in some cases now defunct) businesses. For example, this highly specific image, \includegraphics[scale = 0.275]{RN_6694984_drawing}, evocative of the Aztec cosmology stone, was registered in 2022 for restaurant services by a small Chicago restaurant, Tacos Lotería, that seems now to have gone out of business.\footnote{Tacos Lotería means ``Lottery Tacos'' in Spanish. \textit{Did Tacos Loteria Close?}, \textsc{Reddit}, \url{https://www.reddit.com/r/LoganSquare/comments/1ev1nfm/did_tacos_loteria_close}.}

It is perhaps unsurprising these nonword marks were registered so recently. Today's aesthetic environment overwhelmingly emphasizes these types of brand indicators. Marketing professionals now routinely advise that a name or logo, while necessary, is no longer sufficient to create a recognizable, powerful brand. Instead, in addition to basic branding, businesses must carefully steward their overall ``brand image,'' their ``visual identity,'' and the ``look and feel'' associated with their products.\footnote{Solomon Thimothy, \textit{Why Brand Image Matters More Than You Think}, \textsc{Forbes} (Oct. 31, 2016), \url{https://www.forbes.com/sites/forbesagencycouncil/2016/10/31/why-brand-image-matters-more-than-you-think/} (``Brand image is more than a logo that identifies your business, product or service. Today, it is a mix of the associations consumers make based on every interaction they have with your business.''); Kelly, \textit{supra} note~\ref{supra17}; Katy French, \textit{15 Examples of Brands With a Bold and Beautiful Visual Identity}, \textsc{Column Five}, \url{https://www.columnfivemedia.com/15-examples-of-brand-visual-identity/} (describing various brands' ``visual identities'').} Hence businesses' focus on the unique aesthetics of hard seltzer cans, ginseng packaging, handbag stitching, or urban biological conservatories. The legal effect of this marketing logic has been ``an attempt to expand the purview of intellectual property law into the intangible qualities of spaces as they are experienced by the consumer'' in order to protect those intangible qualities as an abstract source of wealth, rather than as concrete things in themselves.\footnote{\label{supra18} Winnie Won Yin Wong, \textit{Ambience as Property: Experience, Design, and the Legal Expansion of ``Trade Dress''}, 11 \textsc{Future Anterior} 89, 90 (2012); \textit{see also} Thomas D. Dreshcer, \textit{The Transformation and Evolution of Trademarks—From Signals to Symbols to Myth}, 82 \textsc{Trademark Rep.} 301, 340 (1992) (``We are entering an era in which the trademark will not only guide the consumer through the realm of experience, it will to an increasing degree create and define the experience itself.''); \textit{cf.} Bernard Rudden, \textit{Things as Thing and Things as Wealth}, 14 \textsc{Oxford J. Legal Stud.} 81 (1994) (distinguishing, in common-law systems, between two different ways of treating the same assets: treating them as ``thing,'' which subjects those assets to a conventional real property-like regime, and treating them as ``wealth,'' which subjects those assets to a financialized regime that incorporates principles of trusts-and-estates law).} It is no wonder, then, that we are seeing increasing numbers of trademark applications that seek to propertize the nonverbal markers that make up a brand's general aesthetic atmosphere—what we might call a brand's ``aura.'''\footnote{On the phenomenon of brands using intellectual property law to protect a diffuse brand identity that could be characterized as ``auratic,'' see generally  Stefan Bechtold \& Christopher Jon Sprigman, \textit{Intellectual Property and the Manufacture of Aura}, 36 \textsc{Harv. J.L. \& Tech.} 291 (2023).} For reasons already discussed, such highly specific applications tend to involve a substantial number of visual elements, reflected in a high number of design search codes.

But this trend reveals an even deeper shift in marketing logic. The modern usage of the word ``brand'' to mean ``business image'' points to an ancient lineage. Commercial markings may have initially manifested themselves in antiquity as designs branded on livestock, a practice that developed in parallel with the culturally widespread practice of placing a name or identifying symbol on goods of which one is the maker or for which one is otherwise responsible.\footnote{\label{supra19} \textit{See} Sidney A. Diamond, \textit{The Historical Development of Trademarks}, 65 \textsc{Trademark Rep.} 265, 266–67 (1975) (``In all likelihood, the first kind of marking was the branding of cattle and other animals.''); William Safire, \textit{Brand}, \textsc{N.Y. Times} (April 10, 2005), https://www.nytimes.com/2005/04/10/magazine/brand.html (``The burned-in mark, in the 19th century, began to signify ownership not just of an animal but also of liquids in wooden casks, like wine or ale.''); \textit{see also} Matal v. Tam, 137 S. Ct. 1744, 1751 (2017) (``Trademarks and their precursors have ancient origins . . . .''). In noting that the English word ``brand'' etymologically derives from a Germanic verb for the act of burning, we should be careful not to elide the significant diversity of marking practices that preceded modern trademarks. \textit{See} Dreshcer, \textit{supra} note~\ref{supra18}, at 309.} Those names and symbols, in turn, became the first trademarks.\footnote{\textit{See} Diamond, \textit{supra} note~\ref{supra19}, at 270, 280.} As the recent Supreme Court opinion in \textit{Elster} recites, ``even as late as 1860 the term `trademark' really denoted only the name of the manufacturer.''\footnote{Vidal v. Elster, 602 U.S. 286, 305 (2024).} This association between names and trademarks has remained conceptually foundational. Even though the Supreme Court has clarified that the Lanham Act's definition of a trademark permits ``almost anything at all that is capable of carrying meaning'' to act as a mark, the Lanham Act still requires any mark ``to indicate the source of the goods.''\footnote{Qualitex Co. v. Jacobson Prods. Co., 514 U.S. 159, 162 (1995) (quoting 15 U.S.C. \S 1127).} That is, no matter how estranged from the domain of words, any mark should still function like a name, pointing to some specific source. Yet high-code marks erode that requirement even as they seem to satisfy it. Although (as discussed in the previous Section) trademark doctrine is likely to associate increased visual complexity with distinctiveness, many high-code marks do not indicate source in the same way as a conventional word mark or even a conventional nonword logo does. Instead, they are shibboleths. For consumers familiar with the conventions of the relevant industry, high-code marks are often instantly recognizable, even if the consumer does not know the specific brand the mark points to. For other consumers, however, the same nonword mark is just an object. 

A few examples will help. Consider again the highly complex yet subtle background pattern (29 design search codes) used on boxes containing Caperdonich whisky, which is currently marketed by French beverage conglomerate Pernod Ricard:\footnote{The application for this mark is owned by Chivas Holdings (IP) Limited, a UK private limited company that is owned (through two intermediate UK private limited companies) by Pernod Ricard, which is a French entity. \textit{See Search the Register}, \textsc{GOV.UK: Companies House}, \url{https://find-and-update.company-information.service.gov.uk/}.}

\begin{figure}[H]
\centering
\begin{subfigure}[h]{0.2\linewidth}
\includegraphics[width = \linewidth]{SN_98056114_drawing} \
\caption{Drawing of mark submitted to USPTO}
\end{subfigure}
\hspace{30pt}
\begin{subfigure}[h]{0.225\linewidth}
\includegraphics[width = \linewidth]{SN_98056114_specimen} \
\caption{Specimen of mark use}
\end{subfigure}
\caption*{Serial No. 98,056,114}
\end{figure}
\par

\noindent Caperdonich whisky is special. The Caperdonich distillery, which stood near the River Spey in Scotland, is now a so-called ``ghost distillery.''\footnote{\textit{See, e.g.}, Brad Japhe, \textit{Whisky's Ghost Distilleries Are Rising From the Grave}, \textsc{Bloomberg} (June 20, 2024), \url{https://www.bloomberg.com/news/newsletters/2024-06-20/scotch-whisky-ghost-distilleries-are-rising-from-the-grave}.} It was demolished after experiencing repeated business slumps, and consequently, no additional bottles of Caperdonich whisky can be made.\footnote{\label{supra20} Billy Abbott, \textit{Caperdonich—A Lost Speysider with a Smoky Secret}, \textsc{Whisky Exchange: Whisky \& Fine Spirits Blog}, \url{https://blog.thewhiskyexchange.com/2022/07/caperdonich-18-peated-lost-speyside-distillery/}.} This makes them scarce. Additionally, the now-rare Caperdonich has garnered positive reviews by the whisky enthusiasts who have purchased bottles of it, which is perhaps unsurprising (as one might cynically expect these enthusiasts to be motivated to inflate the value of any unopened Caperdonich bottles they own).\footnote{Abbott, \textit{supra} note~\ref{supra20} (praising Caperdonich 18); Thijs Klaverstijn, \textit{Caperdonich 1992 30 Years (Sansibar)}, \textsc{Words of Whisky}, \url{https://wordsofwhisky.com/caperdonich-1992-30-years-sansibar/} (praising Caperdonich 30); Jonah Flicker, \textit{Six Rare Whiskies from Ghost Distilleries to Pay Attention to}, \textsc{Sotheby's}, https://www.sothebys.com/en/articles/six-rare-whiskies-from-ghost-distilleries-to-pay-attention-to (characterizing Caperdonich as ``generally well regarded by whisky fans''). Likewise, one should probably understand Pernod Ricard's revival of Caperdonich as an attempt to liquidate its backstock of a historically unpopular whisky by reframing that backstock as a luxury item.} Thus, prices for Caperdonich have climbed, reaching around \$12,800 for a fifty-year-old malt.\footnote{\textit{Caperdonich 1969}, \textsc{Whisky Exchange}, \url{https://www.thewhiskyexchange.com/p/67203/caperdonich-1969-50-year-old-duncan-taylor-rarest-of-the-rare}.} But although Caperdonich may be favorably regarded by whisky connoisseurs—and although this mark passed the \textit{Seabrook} test—it is hard to imagine that this subdued, intricate design would have inherent meaning for those outside the specialized world of fine whisky. Rather, if anything, this pattern seems like a subtle brand cue that is part of the overall Caperdonich ``look and feel,'' which only a small coterie of consumers has access to and can appreciate. Indeed, luxury markets are a magnet for these sorts of marks. For example, the clasp on the ``Réjane Nano'' handbag (8 codes) referred to in the Introduction,\footnote{Registration No. 5,139,501.}, and the background design on a bottle of Moët \& Chandon ``Ice Impérial'' champagne (9 codes),\footnote{Registration No. 5,860,086.} both passed the \textit{Seabrook} test. Yet it's not obvious that these marks have inherent meaning to consumers who are not already steeped in a luxury \textit{mise-en-scène}.

In this sense, the recent (and perhaps now passé)\footnote{\url{https://www.thefashionlaw.com/quiet-luxury-inches-out-of-fashion-as-does-the-desire-for-handbags/}} trend of ``stealth wealth'' or ``quiet luxury'' epitomizes this use of trademarks. In the early twenty-first century, it was de rigueur for fashion brands to emblazon recognizable brand names and logos on their wares. As scholars recognized, this approach to branding (often referred to as ``Logomania'') represented an attempt by businesses to protect their capital investments in garments that, unless paired with conventional brand indicators like logos, would not receive intellectual property protection.\footnote{Hemphill \& Suk.} But aesthetic mores shifted. And as trademark law seemed increasingly hospitable to more subdued nonword marks of the kind discussed above, brands—including, but not only, luxury fashion brands—embraced an aesthetic of minimalism. Within that aesthetic framework, businesses increasingly communicated their brands to consumers through subtler flourishes and accents. Thus, the fashion brands most centrally associated with stealth wealth, such as Loro Piana and Hermès, aggressively protected what they regarded as their trademarks in the specific shapes or structural details of their products, which were relatively understated and devoid of large logos.\footnote{\url{https://www.thefashionlaw.com/think-trademark-law-doesnt-protect-quiet-luxury-think-again/}} But, as we have seen, understatedness doesn't necessarily translate to visual simplicity. For example, Hermès's mark registration for just the clasp on its ``understated'' Rivale bracelet resulted in an assignment of five design search codes.\footnote{Reg. No. 4725596; \url{https://www.sothebys.com/en/articles/top-7-hermes-bracelets-you-should-be-collecting-now}} Though seemingly paradoxical, this is consistent with this Article's empirical findings. Against a background of increasing minimalism, the desire for subtle but memorable branding elements that stand out may push businesses toward marks whose extreme specificity requires a more-than-average amount of design search codes to describe.

This phenomenon reaches beyond fashion. There are examples across industries of businesses seeking and obtaining trademark rights in detailed agglomerations of shapes that would seem to have little inherent meaning to most consumers, even if they are uncommon or unique enough to pass \textit{Seabrook}. In a case questioned by trademark law scholars,\footnote{Lemley \& McKenna, \textit{supra} note [X].} the Federal Circuit permitted the registration of the background pattern \includegraphics[scale = 0.20]{SN_86269096_drawing} (6 design search codes),\footnote{Serial No. 86,269,096.} which industrial manufacturer Forney Industries used on the packaging for the tools it sold. However, it seems doctrinally and practically unclear that consumers would find inherent meaning in this background image, unless they are familiar with the nuances of hardware packaging.

To be clear, my point is not that the USPTO should only approve such marks if there is evidence that they have acquired distinctiveness—i.e., that the consuming public, in relevant part, recognizes them as marks. My point is that, to function, these marks seem to require a level of sophistication so high it becomes connoisseurship. Indeed, my criticisms here apply with equal force to highly complex marks that the USPTO has registered upon a showing of acquired distinctiveness. Take, for example, the curved contour of the top surface of GROUND ZERO-brand kicking tees for American footballs (10 design search codes), which has been registered as having acquired distinctiveness:

\begin{figure}[H]
\centering
\begin{subfigure}[h]{0.2\linewidth}
\includegraphics[width = \linewidth]{RN_4375439_drawing} \
\caption{Drawing of mark}
\end{subfigure}
\hspace{30pt}
\begin{subfigure}[h]{0.3\linewidth}
\includegraphics[width = \linewidth]{RN_4375439_specimen} \
\caption{Specimen of mark use}
\end{subfigure}
\caption*{Registration No. 4,375,439}
\end{figure}
\par

\noindent Sure, consumers who are attentive to aesthetic details—or who routinely encounter sports paraphernalia—may recognize this exceedingly specific product feature, just as they may recognize the subtle branding on a box containing a Caperdonich bottle. But does some consumers' attentiveness to small, hyper-specific product details mean that those details function like traditional trademarks, like \textit{names} that tell us who makes a product? I think the answer is no. Whereas the traditional mark—the name—is a signifier accessible to all consumers, many visually complex marks are passwords that only the initiated can recognize.

But these password-marks are not just a curious result of doctrinal mutation. They are in tension with fundamental trademark-law concepts, with troubling implications for the relationship between brands and consumers. Under a conventional view of trademark law, the key requirement that a trademark be distinctive is intended to reduce the cost to consumers of identifying the products they like. Instead of verifying a product's quality or attributes each time they buy it, a consumer can simply rely on a mark to tell them who makes the product and thus the quality or attributes they can expect.\footnote{Landes \& Posner.} To do that, a mark must be distinctive—i.e., the consumer must recognize it. And that means \textit{any} consumer. True, some areas of trademark doctrine suggest that mark owners need only demonstrate that their marks are recognizable to the consumers that are likely to buy their products, but those are largely evidentiary rules engineered to deal with technical issues relating to proving secondary meaning.\footnote{\textsc{McCarthy}, \textit{supra} note~\ref{supra5}, \S 15:46.} Distinctiveness has never been an exclusionary concept, because, to serve its economic purpose, it cannot be. 

The new visual complexity of nonword trademarks flips that purpose on its head. Marks like the subtle Caperdonich box pattern above are not an attempt to reduce consumer search costs by helping consumers identify the box's contents as coming from a particular distillery. Rather, such marks are facets of a ``brand image'' or the ``look and feel'' of a product, which may have inherent meaning to the product's core audience but may be meaningless to others. This ``if you know, you know'' approach to trademarks necessarily excludes some consumers to cater to others. 

This is pernicious in two ways. First, when the marks are for highly expensive or luxury products, use of marks as passwords reinforces and intensifies patterns of social exclusion by creating two distinct aesthetic worlds: one for those on the high end of the income distribution and another for everyone else. The result is to ossify the ever-widening gap between those two populations by making the ability to recognize certain branding elements a marker of in-group membership.

Second, and more broadly, using visually complex marks as passwords is wasteful. Against the background of the growing number of applications in general, rising numbers of applications for complex nonword marks require incrementally more resources from the USPTO. Design coding requires significant time and infrastructure that word marks do not, and complex nonword marks may result in a more involved prosecution process.\footnote{Particularly if the proposed mark initially receives an office action objecting that the proposed mark is nondistinctive product design.} Why use public resources to prosecute and maintain marks that, while technically distinctive, are only tenuously related to trademark's policy goals? True, some highly complex nonword marks still serve trademark purposes. For example, given its prominent use of a logo, many consumers will likely recognize the White Claw trade dress above as source-indicative, even if the other, non-logo features claimed in the mark are hyper-specific, verging on esoteric (e.g., the brushstroke-like marking across the front of the can). But it is far less clear that a broad swath of consumers—including those who have no prior familiarity with the relevant products—will recognize marks like the Réjane clasp or the multicolored background on Forney Industries' packaging.

At worst, highly complex nonword marks pose a serious risk of ``trademark clutter'': marks on the register that take a large amount of visual elements out of the public domain without any benefit to consumers, because they are not being used (and therefore cannot help consumers recognize any brand). Consider the Tacos Lotería mark above (\includegraphics[scale = 0.275]{RN_6694984_drawing}). Though it is unfortunate that Tacos Lotería closed, its image mark—which corresponds to 34 codes across 10 different categories and uses as its overall design a relatively common image, the Aztec cosmology stone—will now be on the principal register until 2028, even though no one is using it. As a legitimate business, Tacos Lotería had all the right in the world to seek registration for this mark when it was in use. But opportunistic and even fraudulent trademark applicants, whose businesses may be nonexistent, may be drawn to highly complex nonword marks for the same reason they are drawn to nonsense words of the type so often used by those seeking inclusion in Amazon's Brand Registry.\footnote{Beebe \& Fromer (Fake Trademark Specimens); Fromer \& McKenna.} Just as trademark law is hardwired to disproportionately approve nonsense words (which are necessarily neologisms and are thus fanciful and inherently distinctive), it is also hardwired to disproportionately approve high-code marks for the reasons discussed above.

\section{Policy Implications} \label{sec:3}

\subsection{Distinctiveness as Complexity}\label{subsec:3A}

[To come.]

\subsection{Incorporating Consumer Sophistication into the Distinctiveness Analysis}\label{subsec:3B}

[To come.]

\addcontentsline{toc}{section}{Conclusion}
\section*{Conclusion}

[To come.]

\newpage

\appendix
\addcontentsline{toc}{section}{Appendix A: Dataset}
\section*{Appendix A: Dataset}

This Appendix provides some high-level background on the Trademark Case Files dataset the USPTO makes publicly available, along with a description of how the data I use in Part~\ref{sec:2}'s analysis is derived from that dataset.

The Trademark Case Files dataset is a set of files covering various aspects of the USPTO's trademark application records. Since 2010, the USPTO has made available bulk downloads of data from those records, and, since at least 2013, the USPTO has also offered some form of application programming interface (API) permitting retrieval of that data.\footnote{\label{supra21} \textit{See} Deborah R. Gerhardt \& Jon P. McClanahan, \textit{Do Trademark Lawyers Matter?}, 16 \textsc{Stan. Tech. L. Rev.} 583, 592 (2013); Beebe, \textit{supra} note~\ref{supra8}, at 759–800; \textit{TSDR Data API}, USPTO, \url{https://developer.uspto.gov/api-catalog/tsdr-data-api}.} Bulk downloads of application data are available in two forms: (1) the ``daily XML'' files, in extensible markup language (XML) format, and (2) the ``Trademark Case Files'' dataset, in .dta and .csv formats. The XML files contain the most comprehensive range of data fields and, as the name implies, are updated daily. In contrast, the Trademark Case Files dataset is optimized for large-scale data analysis and is updated annually.\footnote{\label{supra22} \textit{See} Stuart J.H. Graham, Galen Hancock, Alan C. Marco \& Amanda Fila Myers, \textit{The USPTO Trademark Case Files Dataset: Descriptions, Lessons, and Insights}, 22 \textsc{J. Econ. \& Mgmt. Strategy} 669, 670 (2013); \textit{Trademark Case Files Dataset}, USPTO, \url{https://www.uspto.gov/ip-policy/economic-research/research-datasets/trademark-case-files-dataset} (.dta and .csv); \textit{Trademark Full Text XML Data (No Images) - Trademark Trial and Appeal Board (TTAB) (1951 - Present)}, USPTO, \url{https://developer.uspto.gov/product/trademark-daily-xml-file-tdxf-trademark-trial-and-appeal-board-ttab}.} The scope of the Trademark Case Files dataset covers all applications in the USPTO's records as of at least the end of the year preceding the year of release. Thus, my empirical analysis uses the latest iteration of the USPTO's Trademark Case Files dataset, which was released in April 2024 and covers applications filed up to and including March 2024.\footnote{\textit{See} Deborah R. Gerhardt \& Jon J. Lee, \textit{A Tale of Four Decades: Lessons from USPTO Trademark Prosecution Data}, 112 \textsc{Trademark Rep.} 865, 878 (2022) (referring to the .dta- and .csv-formatted files as ``significantly more streamlined''); Graham et al., \textit{supra} note~\ref{supra22}, at 670 (noting that the XML files ``are not well suited to large-scale, comprehensive analysis''); \textit{Updated Trademark Datasets Now Available}, USPTO (Apr. 3, 2024), \url{https://www.uspto.gov/subscription-center/2024/updated-trademark-datasets-now-available} (announcing availability of the .csv files I use here).} Cumulatively, this dataset contains 175 data fields for data relating to about 12.7 million trademark applications, about 7.2 million of which have been registered.\footnote{\textsc{U.S. Pat. \& Trademark Off., USPTO Trademark Case Files 2023 Variable Tables}, \url{https://www.uspto.gov/sites/default/files/documents/casefiles_vartable_2023_final.pdf}.}

Within the Trademark Case Files dataset, my analysis looks only at trademark applications filed in 1986 or after, even though the full dataset includes marks that were registered as early as October 1870. This is for two reasons, both related to the process by which the USPTO digitized its records in the '80s. First, the USPTO's dataset appears to omit a large amount of the trademark applications filed before 1981.\footnote{Beebe, \textit{supra} note~\ref{supra8}, at 760 (observing that ``the dataset does not appear to cover both accepted and rejected trademark applications for the full extent of the 127-year period from 1884 to 2010'' and focusing on an analysis of data from 1981 through 2010).} This makes sense in context. As mentioned in Section~\ref{sec:1}.\ref{subsec:1C}, when the USPTO digitized its trademark records in the '80s, the USPTO only digitized registrations and applications that were ``active'' as of 1983. Thus, although the dataset does contain 41,749 unregistered applications filed before 1983 (3,675 of which were abandoned before 1983), many registrations and applications that were abandoned or otherwise inactive as of 1983 have apparently been lost. Second, 1985 is the earliest year in which the USPTO made publicly available a system for searching digitized trademark records.\footnote{Proposed Plan for an Electronic Public Search Facility, 67 Fed. Reg. 17055 (Apr. 9, 2002) (``The first automated search systems were publicly deployed in 1985 for U.S. trademarks . . . .''). This search system appears to have been internally available at the USPTO in a more limited capacity beginning in 1984. \textit{See Hearing Before the H. Subcomm. on Courts, Civil Liberties, and the Administration of Justice of the Committee on the Judiciary}, 99th Cong. 4 (1985) (``[The USPTO] began using the [first digitized trademark search] system on a limited basis for searching and retrieving the word portions of trademarks in July 1984.''); Alice H. Glasgow, \textit{Trademark Office: Progress Toward Automation}, 32 \textsc{Fed. Bar News \& J.} 226 (1985) (noting ``the implementation in late 1984 of an online trademark search system'' but indicating that this search system had not been made publicly available).} Thus, although the USPTO's dataset contains hundreds of thousands of applications with pre-1986 filing dates, 1986 is the earliest year for which we can be confident that all newly received trademark applications were digitized, regardless of whether they ultimately matured to registration.\footnote{A less conservative position would be to include data from 1985 in my analysis, because there are some indications that the USPTO had already begun digitizing all newly received applications as of 1984, even if those applications did not ultimately mature to registration. \textit{See} J. Howard Bryant, \textit{USPTO's Automated Trademark Search System}, 9 \textsc{World Pat. Info.} 5, 5 (1987) (``The T-Search database contains records representing pending applications and registered trademarks, both those in force and those abandoned, cancelled or expired after 5 March, 1984.''). I opt to err on the side of caution.} This approach is consistent with other scholars' use of this dataset.\footnote{\label{supra23} \textit{See, e.g.}, Deborah R. Gerhardt \& Jon J. Lee, \textit{Sound Marks}, 108 \textsc{Minn. L. Rev.} 2339, 2393 (2024) (beginning analysis in 1981) [hereinafter \textit{Sound Marks}]; Deborah R. Gerhardt \& Jon McClanahan Lee, \textit{Owning Colors}, 40 \textsc{Cardozo L. Rev.} 2483, 2521 (2019) (beginning analysis in 1987) [hereinafter \textit{Owning Colors}]; Beebe \& Fromer, \textit{supra} note~\ref{supra4}, at 973 n.132 (beginning analysis in 1985); Gerhardt \& McClanahan, \textit{supra} note~\ref{supra21}, at 593 (beginning analysis in 1984). [NTD: Add cites.]} Further, although the USPTO's data comprises applications filed in the first quarter of 2024, I omit any applications filed after December 31, 2023, because my analysis relies on annual counts, and the data related to applications in 2024 is necessarily incomplete at present.

Trimming the full dataset down like this leaves us with data on about 10.7 million trademark applications. Due to both the omitted data discussed above and the significant increase in the rate of new trademark applications in recent decades (see Figure~\ref{fig:1} above), this figure is only about 2 million fewer than the full amount of applications in the dataset—despite omitting 116 years from the 154 years the full dataset covers. Within those 10.7 million applications, roughly 2.3 million contain visual elements and thus have been assigned design search codes.

Within those 2.3 million-odd applications, my analysis bases its core conclusions solely on trademarks that the USPTO has categorized as having no letters or words,\footnote{Such marks are those to which the USPTO gives a ``mark drawing code'' beginning with the digit 2. [NTD: To expand on the issue of marks composed of non-Roman characters being included here.]} which, as noted in Part~\ref{sec:2}, comprises 332,568 applications. This is about 14\% of the total number of all applications with visual elements. I omit marks that include language in combination with visual elements (e.g., \includegraphics[scale = 0.15]{RN_2052508_drawing}) given this Article's focus on nonlinguistic elements of marks. Marks that make no use of language at all provide a clearer window into those elements, particularly because I rely solely on design search codes to indicate a mark's visual complexity. Developing metrics that could integrate the relative complexity of words—which, of course, receive no design search codes—into that analysis may be a worthwhile avenue for future research, but it is beyond the scope of this Article.

\end{document}